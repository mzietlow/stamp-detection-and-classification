% Ignorierte Warnings
\RequirePackage{silence}
\WarningFilter{scrbook}{Usage of package `fancyhdr'}  % Hepthesis issues
\WarningFilter{scrbook}{Usage of package `tocbibind'} % Hepthesis issues
\WarningFilter{biblatex}{File }  % Bei Gelegenheit gegenprüfen!
\WarningFilter{tocbasic}{`tocbibind' redefinition of `\listoffigures'}
\WarningFilter{tocbasic}{`tocbibind' redefinition of `\listoftables'} 
\WarningFilter{chktex}{You should perhaps use `\min' instead.} % unfortunately -
\WarningFilter{chktex}{You should perhaps use `\max' instead.} % - doesn't work
\WarningFilter{latex}{Marginpar on page} % due to todonotes


\documentclass[12pt,oneside,a4paper]{hepthesis}

% Language-Encoding und Font
\usepackage{polyglossia}        % Alternative zu Babel
\setmainlanguage[variant=british]{english}
\setotherlanguage[babelshorthands=true]{german}
\addtokomafont{labelinglabel}{\sffamily} % ???

% Citation, Verweise
\usepackage[style=ieee,backend=biber]{biblatex}
\addbibresource{BibLaTex/citation_db.bib}
\usepackage{listings}           % refer to 'minted vs. texments vs. verbments'
\usepackage{hyperref}           % Verlinkungen im Dokument
\usepackage{nameref}
\usepackage{booktabs}           % Tabellenpaket-keine vertikal und dicken Linien
\usepackage[inline,shortlabels]{enumitem}
\usepackage[acronym]{glossaries}		% \gls{<golssary-entry>}
\setacronymstyle{long-short}

% Fonts
\usepackage{fontspec}
\usepackage{unicode-math} % fontspec wird auch von unicode-math geladen?
\setmainfont{TeX Gyre Pagella}
\setmathfont[ItalicFont=*, BoldFont=*]{TeX Gyre Pagella Math}

% Math-Packages und Fonts
\usepackage{physics} % ||norm|| und |abs|
\usepackage{amsmath}

% Formatierung
\usepackage{geometry}           % Paket für Änderung der Seitenränder
\usepackage{setspace}           % Paket für Zeilenabstand 
\spacing{1.25}                  % Zeilenabstand setzen
\geometry{a4paper,left=20mm,right=20mm, top=20mm, bottom=20mm, headsep=7mm}
\raggedbottom{}
\usepackage{microtype}

% Subsections
\setcounter{secnumdepth}{5}
\setcounter{tocdepth}{5}


% Fileinsertion
\usepackage{pdfpages}           % Paket zum Einfügen von PDFs

% Tabellen etc
\KOMAoption{numbers}{noenddot}  % Keine End-Punkte (4., 4.1.) in Nummerierungen?

% Sonstiges                     % Einfügen von Grafiken
\usepackage{graphicx}
\usepackage{subfig}
\graphicspath{ {./bilder/} }
\usepackage{xcolor}
\usepackage[normalem]{ulem}     % Unterstreichen von Text mit \uline
\usepackage{titling}  
\usepackage[de-DE]{datetime2}   % Kalenderdaten in deutschem Format
\usepackage{blindtext}          % Repräsentativer als Lorem Ipsum
\usepackage{kantlipsum}         % Cooler als blindtext
\usepackage{todonotes} % Einfügen von Todos, Option: [disable]
\usepackage[shortcuts]{extdash} % \=/ for nonbreaking dash
\usepackage{url}
\usepackage{cleveref}
\crefname{const}{constraint}{constraints}
\usepackage{xifthen}
\usepackage{xargs}



%%%% <Workarounds> %%%%
% Für Package:todonotes
\setlength{\marginparwidth}{2cm}

% Fix für unbekannten Fehler
\DeclareOldFontCommand{\bf}{\normalfont\bfseries}{\mathbf}

% stoppt Floats mit \FloatBarrier. Subsections sind NICHT enthalten
% Ein Fix ist mit erheblichem Zeitinvest verbunden.
\usepackage[section]{placeins}

% Math-Bugfixes
\usepackage{lualatex-math}

% fancyhdr und KOMA-Script Klassen sollten nicht gemeinsam genutzt werden
% ich habe noch keine Zeit für einen Fix gefunden. scrhack ist ein quickfix.
\usepackage{scrhack}

%%%% </Workarounds> %%%%



%%%%% Custom Commands %%%%%

% Add labels to description-environment:
% #1 is the label, #2 is the item-name, #3 is an optional name for \ref
% Unfortunately, optional arguments won't work in the description-environment :(
\makeatletter
\newcommand*{\namedlabel}[3]{%
  \begingroup%
    #2%
    \ifthenelse{\isempty{#3}}%
      {\def\@currentlabel{#2}}%
      {\def\@currentlabel{#3}}%
    \phantomsection\label{#1}%
  \endgroup%
}
\makeatother

% ref with both chapter number & chapter name
\newcommand*{\fullref}[1]{\hyperref[{#1}]{\ref*{#1} \nameref*{#1}}} 

% cite the authors name and also the work; fcite stands for fullcite
% #2 is the author, #1 is the optional parameter, e.g. fcite[2]{ttt.} for p. 2
\newcommandx{\fcite}[3][1=,2=]{%
  \ifthenelse{\isempty{#2}}%
    {\citeauthor{#3}~\cite[#1]{#3}}%
    {\citeauthor{#3}~\cite[#1][#2]{#3}}%
} 
% example \fcite[see][15--20\psq]{WojciechSamek.2015} 
% biblatex provides macros for sq. sqq. which are equal to f. ff.
% please see: https://texwelt.de/fragen/14065/seitenzahlsequenz-mit-biblatex-kennzeichnen 
% and also http://ctan.ebinger.cc/tex-archive/macros/latex/contrib/biblatex/doc/biblatex.pdf


%%%%% /Custom Commands %%%%%

% Metadaten %
\makeglossaries{}               % makeglossaries <dateiname> per cmd
\title{A Generic Method for Stamp Detection and Classification using Machine Learning Object Detection Frameworks}
\author{8323}
\date{\today}
\hypersetup{
 pdfauthor = {\theauthor},
 pdftitle = {\thetitle},
 pdfsubject = {Transferleistung, Techniker Krankenkasse/Nordakademie, 2018},
 bookmarksopen=true
}


% Eigene Variablen und Commands %
\newcommand{\theLocationAndDate}{Hamburg, den \thedate}


\begin{document}

\setacronymstyle{long-short}

% ----------------------------- Acronyms -----------------------------
% \glslongpluralkey and glsshortpluralkey for plural. might night work betimes
% deleting _all_ auxiliary files _might_ help.
\newacronym{ml} {ML} {Machine Learning}
\newacronym{skcm} {SKCM} {Simple K-Counting Machine}
\newacronym{ai} {AI} {Artificial Intelligence}
\newacronym{rnn} {RNN} {Recurrent Neural Network}
\newacronym{cnn} {CNN} {Convolutional Neural Network}
\newacronym{fcnn} {FCNN} {Fully Convolutional Neural Network}
\newacronym{yolo} {YOLO} {You Only Look Once}
\newacronym{ssd} {SSD} {Single Shot MultiBox Detector}
\newacronym[description={A common activation function for neural neworks.
\(f(x)=\max(0, x)\)}] {relu} {ReLU} {Rectified Linear Unit}
\newacronym{map} {mAP} {mean Average Precision}
\newacronym{svm}{SVM}{Support Vector Machine}
\newacronym{fcn}{FCN}{Fully Convolutional Neural Network}
\newacronym{iou}{IoU}{Intersection over Union}
\newacronym{ap}{AP}{Average Precision}
\newacronym[\glslongpluralkey={Bounding Boxes}, \glsshortpluralkey={bboxes}]{bbox}{bbox}{Bounding Box}
\newacronym{fp}{FP}{False Positive}
\newacronym{tp}{TP}{True Positive}
\newacronym{fn}{FN}{False Negative}
\newacronym{tn}{TN}{True Negative}

\newacronym[description={Umsetzung eines Features oder Produktes mit zwar
minimalem Funktionsumfang, aber dennoch konkretem Mehrwert für den Nutzer},
plural={MVPs}] {mvp}{MVP}{Minimal Viable Product}

% ----------------------------- Glossary Entries -----------------------------

\newglossaryentry{feature map} {
    name={feature map},
    description={Output of a convolutional layer after applying an activation function like \gls{relu}}
}

\newglossaryentry{coco} {
    name={COCO},
    description={Common Objects in Context (COCO) is a well known dataset for
    object detection, segmentation and captioning dataset}
}

\newglossaryentry{voc} {
    name={Pascal VOC},
    description={Pascal VOC is a well known dataset/challenge object detection
    and classification}
}

\newglossaryentry{oi} {
    name={Open Images},
    description={Google Open Images is a well known dataset/challenge for
    object detection and classification}
}

\newglossaryentry{anchor} {
    name={anchor},
    description={Center of a bounding box},
}

\newglossaryentry{layer} {
    name={layer},
    description={Contextsensitive, either dense- or \gls{convolutional layer} 
    as defined by in}}

\newglossaryentry{convolutional layer} {
    name={convolutional layer},
    description={Composition of the convolutional operation and nonlinearity, 
    partially referring to the `complex layer terminology' 
    in~\cite[341]{Goodfellow.2016}}
}

\newglossaryentry{dense layer} {
    name={dense layer},
    description={Also: fully connected layer. Composition of the matrix operation and nonlinearity, partially referring to the `complex layer terminology' 
    in~\cite[341]{Goodfellow.2016}} \todo{refine.~explain also conv in appendix.}
}

\newglossaryentry{deconvolutional layer} {
    name={deconvolutional layer},
    description={Deconvolutional layer. Composition of the deconvolutional 
    operation and nonlinearity, partially referring to the `complex layer 
    terminology' 
    in~\cite[341]{Goodfellow.2016}} \todo{refine.~explain also conv in 
    appendix.}
}

\includepdf[pages=1]{misc/deckblattTerminiert.pdf}

\begin{frontmatter}  %% Deckblatt, Abstract, diverse Verzeichnisse, Glossar.

  \listoftodos{}            % Aktuelle ToDo-Notes
  \newpage
  \tableofcontents        % Inhaltsverzeichnis
  \printglossaries{}      % Abkürzungsverzeichnis/Glossar
  \newpage
  \thispagestyle{empty}
  \addtocounter{page}{-4}
\listoftodos{}
\end{frontmatter}

\begin{mainmatter}
\section{Introduction}
Digitalization is a recurring topic in German politics. Especially in governmental
contexts however, formal letter writing is still sometimes a requirement.

This paper is motivated by a use case identified within the author's corporation,
regarding incompletely filled out billing documents. While the respective document
contains rows explicitly for the name and address of the contractor, it is common
practice that contact details are not provided via the fields. Instead, contractors
use an ink stamp. Unfortunately, this practice excludes the document from currently
applied simple automation pipelines. Therefore, a multistage plan has been
forged, to enable automated document processing. Its first stage comprises the
detection of a \gls{bbox} around the stamp.

The paper is structured as follows. In \cref{sect:related-work}, related work
on the topic of stamp detection is collected and discussed. In \cref{sect:methodology},
the method that will be applied in this paper is thoroughly discussed. In \cref{sect:exp-setup},
a public dataset (\acrshort{staver}) for stamp detection is introduced, and results
are discussed. Finally, in \cref{sect:results-and-discussion}, the paper is wrapped
up and future research directions are presented.

Unfortunately, introduction to fundamental concepts from statistical learning is
beyond the scope of this paper. The reader is therefore expected to be roughly
familiar with basic terminology, regarding especially the usage of neural networks.

\section{Related Work}\label{sect:related-work}
In the past, a variety of methods for stamp detection has been proposed.
One patent on stamp detection, for example, dates back as far as
1963~\cite{Steinbuch.1963}.

A typical strategy found in most publications is to first separate a set of
stamp-candidates from text and background, followed by a \textit{verification} of those
candidates by applying thresholds on multiple previously chosen features.

To provide an overview of the topic, recent work on stamp detection is split into
two major- and respective sub-categories in the following.

{\bfseries{}Restricted Approaches}
    \begin{description}[font={\color{red!50!black}\ttfamily}]
        \item[Color restricted]
                \Textcite{Micenkova.2011} present an approach based on color
                clustering and geometric features. To extract a set of
                stamp-candidates from a given image, they assume stamps to
                be chromatic objects. In consequence, all achromatic parts
                are dropped off the image. To separate the remaining candidates
                for individual analysis, the XY-Cut algorithm proposed
                in~\cite{Krishnamoorthy.1993} is applied. For classification,
                an ensemble of geometric- (pixel density, skew, etc.) and
                color-related features (standard deviation of hue) is compared
                with empirically set thresholds. By treating stamps as
                chromatic objects, they are unable to detect black stamps
                and face problems when colored fonts are used.                
                In~\cite{Micenkova.2015}, \citeauthor*{Micenkova.2015} improve
                upon this method by applying additional geometric features.
                
                A predecessor of the approach by \textcite{Micenkova.2011}
                is~\cite{Forczmanski.2010} which uses fewer features for
                \textit{verification} and is only applicable to blue or red
                colored stamps.
            \item[Geometrically restricted]
                An early attempt of using geometric analysis was formulated by
                \textcite{Zhu.2006}, proposing an ellipse detection algorithm
                inspired by Hough transform~\cite{Duda.1972}. However, this method
                is only applicable to elliptical stamps.
    \end{description}
\bigskip{}
{\bfseries{}Generic Approaches}
\begin{description}[font={\color{red!50!black}\ttfamily}]
    \item[Geometric features]
    A more versatile method than~\cite{Zhu.2006} was proposed in
    \cite{Ahmed.2013, Ahmed.2016}.
    \citeauthor*{Ahmed.2013} make use of advances in research of \textit{feature detectors}
    and \textit{feature descriptors}. In a first step, the image is
    binarized~\cite{Tensmeyer.2020} and connected components are extracted.
    For every connected component, \textit{feature key points} and \textit{feature descriptors}
    are computed via FAST~\cite{Rosten.2005} and ORB~\cite{Rublee.2011}.
    Further, \gls{bbox} height and width of each connected component
    are extracted. \textit{Verification} is performed by comparing the extracted
    features of each connected component to a validation set~\cite[120\psq]{Goodfellow.2016}.
    However, the authors notice low precision- and recall-rates, caused by their inability
    to recognize severely overlapping stamps.

    In~\cite{Nandedkar.2015b}, \citeauthor*{Nandedkar.2015b} extend an approach
    they suggested earlier in~\cite{Nandedkar.2015}. They analyze spatial
    frequency components in document images, finding text to be the major
    contributor. To segment a given image, they first conceal
    text-components by removing objects with high spatial frequencies and
    further apply a Gaussian filter. To prevent textual stamps from being
    removed, a chromaticity threshold is applied on the removal of spatial
    frequency components. Stamp-candidates are then identified using mean-shift
    segmentation~\cite{Fukunaga.1975}. Small regions are filtered
    with an empirically set threshold, where the largest region is considered
    to be the background. For \textit{verification} of stamp-candidates,
    AlexNet~\cite{Krizhevsky.2012} is used, producing three classes
    (logo, stamp and noise) in the final \gls{dense layer}. Even though unmentioned
    in the paper, the presented approach of filtering spatial frequency
    components faces limitations regarding black textual stamps which might
    turn out to be removed by the spatial frequency filter.
    \item[Mixed features]
    \citeauthor*{Dey.2015} propose an outlier-based
    method in~\cite{Dey.2015}. To segment a given image,
    it is separated in foreground and background using binarization~\cite{Tensmeyer.2020}.
    To retain color information the obtained foreground-values are applied as
    a mask to the original image. Color information is reduced from
    three-dimensional \textit{RGB} color space to a one-dimensional principal
    component using \gls{pca}~\cite{Hotelling.1933}.
    Stamp-candidates are obtained using a connected-components algorithm on
    the first principal component (from \gls{pca}). To filter out \glspl{fp},
    thresholds on features such as stroke width, \gls{bbox} width and
    height are applied. For \textit{verification}, a set of features is extracted and
    validated using additional empirically set thresholds.

    A sliding window approach using the AdaBoost~\cite{Freund.1995} on cascaded
    decision trees has been proposed in~\cite{Forczmanski.2016}. To train 
    AdaBoost, a labeled training set is constructed. For each training 
    sample, Haar-like features~\cite{Viola.2001} are extracted. AdaBoost then
    constructs a set of consecutive decision trees to optimize stamp classification
    of training samples. Each decision tree takes in a Haar-like feature from 
    a fixed position inside the sliding window. Based on the input feature 
    value, the tree decides whether to \textit{accept} or \textit{reject} a
    given window as stamp-candidate. A set of consecutive decision trees
    with such property (\textit{accept} or \textit{completely reject}) is
    then called a \textit{cascade}. In order for a sliding window to become a valid
    stamp-candidate it has to survive such a cascade of decision trees. The
    set of generated candidates is then \textit{verified} using a broader set of features.
    Several algorithms are compared for \textit{verification} including neural
    networks, decision trees, and \glspl{svm}.

    \textcite{Younas.2017} propose using a \gls{fcn}, built upon the VGG-16~\cite{Simonyan.2015}
    network architecture. In this approach, the final \glspl{dense layer}
    of VGG-16 are replaced with a \gls{deconvolutional layer}. By doing this,
    a prediction map is received as output instead of classification scores.
    To reduce training time, transfer learning from the \gls{voc}2011 dataset is
    applied (although without ablation study of the impact of transfer learning).
    However, no preprocessing is mentioned. Using 90\% of their dataset for
    training and 10\% for \textit{verification} they report ``state-of-the-art''
    results, although noticing difficulties with tabular stamps.
\end{description}

Despite numerous research carried out on stamp segmentation, most of the 
approaches examined in this chapter either face limitations with overlapping 
objects~\cite{Nandedkar.2015, Nandedkar.2015b,Forczmanski.2016, Ahmed.2013, Dey.2015, 
Forczmanski.2015, Forczmanski.2016} or logo misclassification~\cite{Ahmed.2013,Dey.2015,Micenkova.2011}.
Some approaches are limited by runtime, operating for more than 13 Seconds per
evaluation~\cite{Ahmed.2016,Nandedkar.2015}.
\par
Besides segmentation, classification of stamps is observed to be a common task 
in literature. Classes are usually either binary (Stamp and non-Stamp), 
object-related (Stamp, Logo, Text, Tables)~\cite{Forczmanski.2016,
Nandedkar.2015, Nandedkar.2015b, Dey.2015},
or shape-dependent (Circle, Ellipse, Square, etc.)~\cite{Forczmanski.2015}.
However, research on classification of similar stamps i.e.\ stamps belonging to
the same shape-class but differing in text and decor, has so far only been
reported on by~\cite{Petej.2013}.
\section{Methodology}\label{chap:Methodology}
To address the challenges identified in~\cref{chap:Related Work}, namely
evaluation-speed and categorization of near-classes, recent advances in machine
learning are considered. Contrary to most approaches in the field which perform
segmentation, this work focuses on detecting \glspl{bbox}. In this section,
details on the applied method is provided. Methods for object detection can coarsely
be divided into two groups: one-staged and two-staged approaches. To reduce complexity,
only a single one-stage approach will be considered in this paper. For details on
two stage detectors confer e.g.\ \cite{Ren.2015}.

\bigskip
{\large{\textbf{Single Shot MultiBox Detector}}}\\
Two common one-stage architectures are
\gls{yolo}~\cite{Redmon.2015, Redmon.2016b, Redmon.2018} and
\gls{ssd}~\cite{Liu.2016}. While \gls{ssd} can be extended to any base network
(e.g. VGG16~\cite{Simonyan.2015}), \gls{yolo} is limited to only Darknet~\cite{Redmon.2016}.
Therefore, \gls{ssd} was chosen over \gls{yolo} for this paper, retaining the
ability to quickly exchange base networks.

Central concepts of \gls{ssd} are described in the following.

\subsection{Default Boxes}\label{subsect:default-boxes}
The most fundamental part of understanding \gls{ssd} is understanding its specific
concept of \glspl{bbox}, default boxes and anchors. In order to reduce the very high
complexity, this subsection is written in a slightly more colloquial manner.

Slightly anticipating and simplifying \cref{subsect:SSD Architecture}, a
\emph{prediction} from \gls{ssd} is the output from a set of \glspl{convolutional layer}
that is chosen from within the network. Output of a single such \gls{layer} is,
first and foremost, just a \gls{feature map} --- while the goal is to find the
coordinates of a \gls{bbox} (and class predictions). 

Consider such a \gls{feature map} with dimensions \(m \times n\). Every pixel
from that \gls{feature map} contains highly condensed information about a specific
region from the original image\footnote{This concept is called the
\emph{receptive field} of \glspl{cnn}, \cite[cf.][331\psq]{Goodfellow.2016}}.
The \gls{feature map} is then processed further, s.t.\ for every such
pixel (and therefore the related region within the original image) receive the
coordinates of a single \gls{bbox} are received (simplified, for details see \cref{subsect:SSD Architecture}).

To the reader, this might be confusing. Receiving \(m\times n\)
single \gls{bbox} predictions for every pixel within the \gls{feature map} for
every chosen layer makes the number of predicted \glspl{bbox} \(\gg\) than the
expected number of \gls{gt} \glspl{bbox}. And it raises a pressing question:\linebreak
\textbf{How is training data constructed for such an unusual architecture?}

This question is answered (partly) by the introduction of \textbf{anchors}.

\paragraph{Anchors}\label{par:anchors}
Every pixel from a \gls{feature map} is related to a region within the original
input image (\emph{receptive field}~\cite[cf.][331\psq]{Goodfellow.2016}).
Furthermore, \glspl{convolutional layer} preserve the spatial structure of a
convolved image~\cite[cf.][335\psqq]{Goodfellow.2016}.

Therefore, every \emph{pixel} from within the feature map can be related
(very easily) to a region within the original image. The center of such region
is called the \textbf{anchor}. The x and y coordinates for every pixel \(p_{ij}\)
can then be calculated using \cref{eq:anch-x,eq:anch-y}. A practical example for an
image \(I\) of height and width \(I_w\times I_h=300\times 300\) and a
\gls{feature map} \(M\) of height and width \(M_w\times M_h=19\times 19\)
is shown in \cref{fig:vgg16-anchors}.

\begin{align}
    x_i&=0.5*\frac{w_I}{w_M} + \sum_{0}^{i-1} \frac{w_I}{w_M}\label{eq:anch-x}\\
    y_j&=0.5*\frac{h_I}{h_M} + \sum_{0}^{j-1} \frac{h_I}{h_M}\label{eq:anch-y}
\end{align}
where:
\begin{conditions}
    x_i & := & x-coordinates of the anchor for pixel \(p_{ij}\)\\
    y_j & := & y-coordinates of the anchor for pixel \(p_{ij}\)\\
    w_I & := & width of input image \(I\)\\
    w_M & := & width of feature map \(M\)\\
    h_I & := & height of input image \(I\)\\
    h_M & := & height of feature map \(M\)
\end{conditions}
\begin{figure}[t!]
    \centering
    \includegraphics[width=0.5\textwidth]{vgg16-19x19}
    \caption{Generated anchors (blue) for an image of dimensions \(300\times 300\)
    and \gls{feature map} of \(19\times 19\).}\label{fig:vgg16-anchors}
\end{figure}
Now, training data could be constructed by assigning every \gls{gt} \gls{bbox} to
its closest anchor (from every chosen layer). An example of this is given in
\cref{eq:train-anchor}. 
\begin{equation}\label{eq:train-anchor}
    \begin{matrix}
        \ldots\\
        \begin{bmatrix}
            \ldots & \ldots & \ldots & \ldots & \ldots\\
            0 & 0 & 0 & 0 & 0\\
            271 & 828 & 182 & 845 & 1\\
            0 & 0 & 0 & 0 & 0\\
            \ldots & \ldots & \ldots & \ldots & \ldots
        \end{bmatrix}\\
        \ldots\\
        \begin{bmatrix}
            \ldots & \ldots & \ldots & \ldots & \ldots\\
            0 & 0 & 0 & 0 & 0\\
            3141 & 592 & 653 & 59 & 1\\
            0 & 0 & 0 & 0 & 0\\
            \ldots & \ldots & \ldots & \ldots & \ldots
        \end{bmatrix}\\
        \ldots
    \end{matrix}
\end{equation}
where the first four rows are x- and y-coordinate, height and width of the respective
\gls{bbox} and the last row is the class (0: \textit{no-object}, 1: \textit{object}).

This approach is flawed in two ways:
\begin{enumerate}
    \item Assigning \gls{gt} \glspl{bbox} to the closest anchor is entirely agnostic
    of the expected size of the receptive field for the different layers. A layer
    with smaller receptive field per pixel would probably not be able to perceive
    larger objects, while a layer with a larger receptive field might overlook
    smaller objects.\label{itm:anchor-flaw1}
    \item The model has no \textit{a priori} knowledge of \emph{position}, of
    \emph{x-} and \emph{y-coordinates}, of \emph{width} and \emph{height}.
    This is especially critical for images with different resolutions.\label{itm:anchor-flaw2}
\end{enumerate}
A possible solution to these flaws are \textbf{default boxes}.

\paragraph{Default Boxes}\label{par:default-boxes}
Default boxes are used to assign spatial representation to anchors. With such a
representation, every \gls{gt} \gls{bbox} can be matched to the set of regions
(related to the receptive fields) that have a large overlap\footnote{The overlap
could then be computed via \gls{iou} for example --- as is done in \cref{eq:bbox-matcher}.}.

As a starting point, so far a set of chosen layers \(L = \left\{l_0, l_1, \ldots, l_{n-1}\right\}\)
and a set of anchors for every such layer \(\mathbf{A} = \left\{A_0, A_1, \ldots, A_{n-1}\right\}\)
were constructed.

Now, although the exact receptive fields of the chosen layers are uncertain, 
we \emph{do} know that their sizes \(\abs*{recept\left(l\right)}\), are strictly
increasing with subsequent layers, i.e.\ \(\abs*{recept\left(l_0\right)} < \ldots < \abs*{recept\left(l_{n-1}\right)}\).

Fortunately, \textcite{Liu.2016} claim that the exact size of the receptive field is of
subordinate importance. Instead of calculating the exact receptive field, they
propose to assign a fixed size to every chosen \gls{layer} per \cref{eq:default-box-size}:
\begin{equation}\label{eq:default-box-size}
    s_k=s_{_\text{min}} + k * \frac{s_{_\text{max}}-s_{_\text{min}}}{n-1}
\end{equation}
Where
\begin{conditions}
    n               &:= & \(\abs*{L}\)\\
    k               &\in & [0, n-1]\\
    s_{_\text{min}} &=& 0.2\\
    s_{_\text{max}} &=& 0.9
\end{conditions}
A default box \(d_{i,j}\) for a pixel \(p_{i,j}\) within a given layer \(l_k\)
can then be computed as given in \cref{eq:dbox}.
\begin{align}
    \begin{split}\label{eq:dbox}
        d_{i,j}^h  &= s_k * h_I\\
        d_{i, j}^w &= s_k * w_I\\
        d_{i,j}^x  &= x_i \text{ (see \cref{eq:anch-x})}\\
        d_{i,j}^y  &= y_j \text{ (see \cref{eq:anch-y})}
    \end{split}
\end{align}
It is now possible to match every \gls{gt} \gls{bbox} to a set of default boxes
(see \cref{eq:bbox-matcher}).
\begin{equation}\label{eq:bbox-matcher}
    match(\text{bbox}, \text{dbox}) =
    \begin{cases}
        1 & \text{if } \text{IoU}\left(\text{bbox, dbox}\right) > 0.5\\
        0 & \text{otherwise}
    \end{cases}
\end{equation}
where:
\begin{conditions}
    \text{bbox} & := & coordinates of the bounding box\\
    \text{dbox} & := & coordinates of the default box
\end{conditions}
Thereby, \hyperref[itm:anchor-flaw1]{\(\left.\text{flaw 1}.\right)\)} from anchor
construction can be considered as solved.

\hyperref[itm:anchor-flaw2]{\(\left.\text{Flaw 2}.\right)\)} can now be tackled
multiple ways. The most obvious would be to convert \gls{gt} \glspl{bbox} and
default boxes into the \textit{percental} space (i.e. \(\left[0, 1\right]\)).
Yet, this again would require the model to learn different semantics per pixel
within a \gls{feature map}. For example, the upper left pixel/region would produce predictions
within \(\left[0,0.2\right]\times \left[0,0.2\right]\), while the lower right
region would produce predictions within \(\left[0.8,1.0\right]\times \left[0.8,1.0\right]\).
This is problematic, because \glspl{convolutional layer} share weights\footnotemark{}
between inputs (i.e. such pixels/regions)~\cite[cf.][564\psqq]{Murphy.2012}.
\footnotetext{In fact, sharing weights between inputs is the key distinguishing
feature between \glspl{convolutional layer} and \glspl{dense layer}.}
Requiring different semantics \emph{between} these pixels introduces additional
complexity to the \glspl{convolutional layer}.

To alleviate this complexity, rather than coordinates, the offset \emph{between}
the coordinates (of \gls{gt} \glspl{bbox} and the related default boxes) are predicted.
This aligns the task between all pixels of a \gls{feature map}.

\Textcite{Liu.2016} introduce a last quirk inspired by the concept of priors
from bayesian statistics~\cite[cf.][165\psqq]{Murphy.2012}. To improve convergence
they choose multiple aspect ratios (i.e.\ 1:1, 1:2, 1:3, 2:1, 3:1) for their
default boxes. Reducing extent of this paper, the reader is referred to \cite{Liu.2016}.

Finally, the sets of default boxes per \gls{layer} are concatenated and further
encoded as described in \cref{append:Concepts of Bounding Box Encoding}, such
that default box

\subsection{Encoding Ground Truth} For training, ground truth \glspl{bbox}
are encoded with respect to the concatenated default boxes as calculated in
\cref{par:default-boxes}. 


\gls{ssd} is supposed to infer offsets to default
boxes (cf. \cref{subsect:SSD Architecture}). To simplify inference only default
boxes with an \gls{iou} greater than 0.5 are chosen (cf. \cref{sect:Intersect Over Union}). For a \gls{bbox}
\(\text{bbox}=\{\text{bbox}_{x_\text{min}}, \text{bbox}_{x_\text{max}}, \text{bbox}_{y_\text{min}}, \text{bbox}_{y_\text{max}}\}\), default box
\(\text{dbox}=\{\text{dbox}_{x_\text{min}}, \text{dbox}_{x_\text{max}}, \text{dbox}_{y_\text{min}}, \text{dbox}_{y_\text{max}}\}\) we compute the
encoded box \(\text{ebox}\) with \cref{eq:encoded box}
\begin{equation}
    \text{ebox}=\{(\text{bbox}_{x_\text{min}}-\text{dbox}_{x_\text{min}}), (\text{bbox}_{x_\text{min}}-\text{dbox}_{x_\text{max}}), (\text{bbox}_{y_\text{min}}-\text{dbox}_{y_\text{max}})\}\label{eq:encoded box}
\end{equation}
We receive a matrix of dimensions \(\sum_{m\in M}{A_m}\times \sum_{m\in M}{k_m*4}\),
where \(M\) is the set fo all \glspl{feature map}, \(A_m\) is the set of all anchors per
\gls{feature map} and \(k_m\) is the amount of aspect ratios per \gls{feature map}. Finally
we concatenate the class label to every encoded \gls{bbox}, such that the matrix
dimensions now are \(\sum_{m\in M}{A_m}\times \sum_{m\in M}{k_m*4+c}\), where c
is the amount of classes. Every row is now in the form of
\begin{equation}
    \left\{x_{_\text{min}}, x_{_\text{max}}, y_{_\text{min}}, y_{_\text{max}}, \text{one\_hot\_classes} \right\}
\end{equation}


\subsection{SSD Architecture}\label{subsect:SSD Architecture}
As mentioned in the introduction to this section, \gls{ssd} is independent of any
specific base network. Exemplary, \cref{fig:ssd-vgg} shows the architecure of
SSD for VGG16~\cite{Simonyan.2015}. Fully connected \glspl{layer} of VGG16 are
dropped and replaced with additional \glspl{convolutional layer}. We denote: 
\(\text{Network}:=\text{Base Network}\rightarrow \text{Additional Layers}\).

Then, a set of \glspl{convolutional layer} \(L\) (blue and yellow in \cref{fig:ssd-vgg})
is chosen from the network (i.e.\ \(L\subseteq \text{Network}\))\footnote{\label{foot:receptive}The motivation
behind choosing multiple layers \(L\) from the network is as follows: the \glspl{feature map}
produced by the \glspl{convolutional layer} get smaller with every additional layer.
Colloquially speaking, the relationship between such a small \gls{feature map} and
the input image is, that one pixel within the small \gls{feature map} is related
to multiple pixels within the original input image. Therefore, when looking at the
\glspl{feature map} from subsequent \glspl{convolutional layer}, one is looking
at information that is extracted from the image at different scales.

Making this logic concrete, in a larger \gls{feature map}, one pixel might contain
condensed information about objects like wheels or headlights, whereas in a smaller
\gls{feature map} one pixel might contain  condensed information about the entire
car. This concept is called the \textit{receptive field}~\cite[cf.][331\psq]{Goodfellow.2016}.}.

To produce class- and \gls{bbox} predictions, every chosen \glspl{layer} \(l\in L\)
is directed into a final additional \gls{convolutional layer} (i.e.\ one additional
\gls{convolutional layer} per \gls{layer} in \(L\), depicted green in \cref{fig:ssd-vgg}).

Das hier setzt halt alles Wissen voraus, das noch gar nicht aufgebaut wurde!

The number of filters for these subsequent \glspl{layer} is chosen as \((4+c)*k\)
with \(c\) classes\footnote{Actually, \(c+1\) classes. This additional class is
the \textit{no-class}-prediction, because obviously most areas within the image
contain no classifiable object.} and \(k\) default boxes per \gls{anchor}.
\begin{figure}[ht]
    \centering
    \includegraphics[width=1\textwidth]{vgg16-ssd}
    \caption[Example of the SSD architecture using VGG16 as its base network]{Example
    of the SSD architecture using VGG16 as its base network~\cite[cf.][]{Liu.2016}.
    \\\\
    Blue colored boxes represent the additional \glspl{convolutional layer} that are added to the base network.
    \\\\
    Green boxes represent the final additional \glspl{convolutional layer} that produce
    the classes- and \gls{bbox} predictions.}
    \label{fig:ssd-vgg}
\end{figure}

\subsection{Training}
\ldots

\subsubsection{Loss Function}
The loss is a composite of two separate loss functions:

\begin{align}
    L_{\text{conf}(x, c)}\\
    L_{\text{loc}(x, l, g)}\\
    L(x,c,l,g) = \frac{1}{N}*\left(L_{\text{conf}(x, c)} + \alpha * L_{\text{loc}(x, l, g)}\right)
\end{align}

\begin{equation}
    \begin{cases}
        L(x,c,l,g) = \frac{1}{N}*\left(L_{\text{conf}(x, c)} + \alpha * L_{\text{loc}(x, l, g)}\right),& \text{if } N \geq 1\\
        0, & \text{otherwise}
    \end{cases}
\end{equation}

\begin{equation}
    L_{\text{loc}(x, l, g)} = \sum_{i \in Pos }^{N}
\end{equation}
where:
\begin{conditions}
    N & := & number of matched boxes\\
    x & := & pixel under consideration\\
    c & := & class scores\\
    l & := & Predicted Boxes\\
    g & := & Ground Truth Boxes\\
    \alpha & := & weighting parameter, increase to put importance on the localization loss.
\end{conditions}

\chapter{Experimental Results}
This chapter gives an overview of the experimental setup, applied metrics
(\cref{sect:metrics}), and data sets (\cref{sect:staver,sect:tkstede}). Summary
and conclusion are given in \cref{sect:results-and-discussion}.

\section{Metrics}\label{sect:metrics}
To evaluate and compare approaches described in \cref{chap:Methodology} we
discuss a set of metrics in this section. We consider
\nameref{subsect:precision-recall-fscores}, \nameref{subsect:mAP} and also
introduce \nameref{subsect:normalized-iou}.

\subsection{Precision, Recall \& F-Scores}\label{subsect:precision-recall-fscores}
To the best of our knowledge, the most influential metric used in stamp 
detection is pixel-wise evaluation of the Precision and Recall
tuple~\cite{Nandedkar.2015121620151219,Younas.2017110920171115,
Ahmed.2013082520130828,Dey.2015121620151219,
Micenkova.2011091820110921, Bhalgat.16.09.2016, Micenkova.2015,
Nandedkar.2015082320150826}. Often, recall and precision are used in settings
with class imbalance where they provide a more sensible measure than accuracy.
In the stamp detection, every image pixel one can consider two classes, either
\textit{stamp} or \textit{non-stamp}. In training and evaluation images,
\textit{non-stamp}-pixels greatly outnumber \textit{stamp}-pixels. Therefore,
showing obvious class imbalance.
\par
A definition of precision and recall is given
in~\cite[423]{Goodfellow.2016} as the fraction of \textit{true} positive
detections over positive detections (i.e.\ true and false positives) and the
fraction of \textit{true} positive detections over positive \textit{datapoints}
(i.e.\ true positives and false negatives):
\begin{align}
    \text{Precision} &= \frac{t_\text{positives}}{t_\text{positives} + f_\text{positives}}\\
    \text{Recall} &= \frac{t_\text{positives}}{t_\text{positives} + f_\text{negatives}}
\end{align}
Intuitively, precision is the ``ability of a classifier to distinguish a
negative sample from [a] positive one''~\cite{Younas.2017110920171115}, while
recall is ``the ability of a classifier to classify all positive samples''~\cite{Younas.2017110920171115}.
A visual example of recall and precision is given in
\cref{fig:visual-precision-recall}.
\begin{figure}
    \center
    \includegraphics[width=\textwidth]{Metrics2.png}
    \caption{Visual example of recall and precision, with groundtruth (red) and
    inferred detection (blue)}\label{fig:visual-precision-recall}
\end{figure}


\subsection{Mean Average Precision}\label{subsect:mAP}
Despite its extensive use in the object detection community~\cite{Liu.2016,Ren.04.06.2015}, 
\Gls{map} was only applied in a single work \cite{Zhu.2006} from 2006. This is,
because usually ranks are not generated in most approaches, therefore the 
underlying metric of average precision, which \textit{is} computed over ranks) 
lacks meaning.

\subsection{Normalized Intersection Over Union}\label{subsect:normalized-iou}
\blindtext[1]
\todo{This kind of has the same issues as accuracy. Imagine, 99\% of images are
in the lower left corner, the a System that learns this would have a high 
average IoU. Maybe combine this from ideas of the F-Score.}

\section{StaVer}\label{sect:staver}
``The presented method is evaluated on a publicly available dataset (StaVer1) for stamp
detection and verication [sic!]. This dataset contains 400 scanned document images. Out
of these 400 documents, 80 documents contain black stamps wheras the remaining 320
documents contain colored stamps. All of these document images are available in 200,
300, and 600 dpi. For each image, two different types of ground truths are available.
One contains the pixel level ground truth, which means all of the pixels which belong to
stamps are marked in the image. The other ground truth format contains bounding box
information for each stamp. Hence, this dataset can be used for both pixel level as well
as patch level evaluation of stamp detection. In addition, it contains different types of
stamps ranging from rectangular, oval, to irregular shaped, and most importantly, textual
stamps.
For evaluation of the presented approach, the training set is generated by using 36 documents
out of 400. Out of these 36 training documents, only 6 contains black stamps
whereas the remaining 30 are with colored stamps. Testing is performed on the remaining
364 documents (74 documents with black stamps, 290 documents with colored stamps).
All the results are reported for 200 dpi documents.''~\cite{Ahmed.2016}

\section{TKStede}\label{sect:tkstede}
\blindtext[1]

\section{Results \& Discussion}\label{sect:results-and-discussion}
Goals: 
    a.) How good can stamp detection get with current frameworks
    b.) How good can classification of a vast variety of similar classes get
        with current frameworks
\chapter{Conclusion}
\blindtext[1]

\section{Future Research}
\blindtext[1]

\section{Lessons learned}
\blindtext[1]
\todo{Mention Buda 2018, A systematic study of the class imbalance problemin convolutional neural networks}
\end{mainmatter}

\begin{appendices} %% Anhänge die nicht zwingend für die Arbeit notwendig sind
\chapter{Basic Concepts}\todo{Exchange for a better title}
\blindtext[1]

\section{Concepts of Bounding Box Encoding}
\blindtext[2]

\section{Preprocessing}
\blindtext[3]

\section{Concepts of Neural Network Architectures}
\blindtext[4]

\section{Loss Functions}
\blindtext[2]

\section{VGG16 and ResNet}
\blindtext[3]


%\chapter{General Todos}
\todo{Keine absoluten Aussagen.}
\todo{Deckblatt anpassen}
\todo{Arab. Nummerierung im Appendix, Backmatter?}
\todo{Double check spatial frequency paper [8, 9]}
\todo{Glossar füllen, PCA etc.}
\todo{extract best practices from dissertation}
\todo{Cite FAST and ORB, Gaussian, mean shift, PCA, K-NN, decision tree, SVM, 
VGG16, FCN, deconvolution, VOC2011?}
\todo{PCA is in math environment\dots}
\todo{Generate a `class prototype' for stamp and no-stamp classes. (what is the
perfect stamp picture)}
\todo{Write a colour shifter for preprocessing, s.t. every colour is trained
and maybe even consider a color gradient shifter}
\todo{Why combine localization and classification, why not split both? Is it speed only?}

\end{appendices}

\begin{backmatter} %% bibliography, tables of figures etc., index...)
\input{Dokument/04_Backmatter}
\end{backmatter}

\end{document}
