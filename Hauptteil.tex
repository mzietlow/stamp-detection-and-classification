\chapter{Einleitung} In der heutigen Hochtechnologiegesellschaft besteht auch
weiterhin ein \mbox{ungebremstes} Interesse daran, immer schnellere, kleinere
und effizientere Prozessoren zu \mbox{entwickeln}. Bisher konnte die
Halbleiterindustrie diesem weitestgehend nachkommen, dennoch zeichnen sich
Grenzen dieser Miniaturisierung ab, die unüberwindbar scheinen. Grundlegende
physikalische Begebenheiten wie Quanteneffekte führen dazu, dass funktionale
Strukturen nicht weiter verkleinert werden können, ohne sie massiv zu
beeinträchtigen und damit unbrauchbar werden zu lassen \cite{Moore.2017}.\\  Um
den technischen Fortschritt auch in Zukunft vorantreiben zu können, bedarf es,
ob der Unumgänglichkeit dieser Effekte, neuer Denkansätze. Einer dieser Ansätze
umfasst die Verwendung optischer Schaltkreise. Diese bieten den Vorteil eines
Informationstransportes in Lichtgeschwindigkeit, im Gegensatz zu der deutlich
langsameren \mbox{Geschwindigkeit} der Elektronen in konventionellen
Schaltkreisen. Auch die Schaltzeiten sind gegenüber den herkömmlichen
elektronischen Bauteilen stark reduziert \cite{Simonite.2010,Johnson.2015}, denn
bei elektronischen Bauteilen ist die Driftgeschwindigkeit der Elektronen ein
limitierender Faktor – dieser entfällt bei optischen Bauteilen. Durch beide
\mbox{Begebenheiten}, der schnelleren Transportgeschwindigkeit und den kürzeren
Schaltzeiten, \mbox{erhöht} sich der Datentransfer gegenüber herkömmlichen
elektronischen Prozessoren drastisch. Die Integrierte Optik (IO) als Technik der
nächsten Generation steckt aber noch in den Kinderschuhen und es wird ausgiebig
an der Verwirklichung geforscht \cite{Touch.2017}, in Erwartung, dass der Markt
in den nächsten Jahren stark wachsen wird \cite{Credence.2017}.\\ Für diese neue
Art des Rechnens wird eine gerichtete kohärente Lichtquelle benötigt – ein
Laser. Halbleiternanodrähte bilden eine solche Klasse miniaturisierter Laser und
könnten in Zukunft diese Lücke füllen. Sie besitzen einen Durchmesser von
wenigen hundert Nanometern, bei einer Länge von wenigen Mikrometern. Aufgrund
eines \mbox{Brechungsindex} größer eins, können sie Licht einer Wellenlänge
führen, die größer ist als ihr eigener Durchmesser \cite{Zimmler.2010}. Damit
setzen Nanodrähte eine untere Grenze für die gerichtete Laseremission,
gleichzeitig fungieren sie aufgrund ihrer Struktur als natürliche Resonatoren
~\cite{Eichhorn.2013}. Sie besitzen Schaltzeiten im Bereich weniger Pikosekunden,
drei Größenordnungen unter den Schaltzeiten moderner Transistoren
\cite{Sidiropoulos.2014,Qiu.2017}. Damit erfüllen Nanodrähte die Anforderungen,
die die Gesellschaft an ein solches System stellt – sie sind schnell, sie sind
klein und sie sind effizient.\\ Für eine letztendliche Anwendung bedarf es
jedoch hinreichender Kenntnis über die Betriebsparameter und über die
Emissionseigenschaften, denn selbige müssen für die Anwendung reproduzierbar
sein, um einen stabilen Betrieb zu ermöglichen – hier setzt diese Masterthesis
an. Kenntnis über die Abstrahlcharakteristik der Nanodrähte ist notwendig, um
das Licht weiterverarbeiten zu können. Hierbei spielt die Polarisation des
Lichtes eine entscheidende Rolle, denn diese lässt sich leicht manipulieren und
eignet sich somit gut für optische Rechenprozesse \cite{Lohmann.1986}.\\ In
dieser Masterthesis wird die Abstrahlcharakteristik von Zinkoxid
(ZnO)-Nanodrähten untersucht. Zinkoxid ist ein Materialsystem mit einer
Lichtemission im nahen UV-Bereich bei ca. 3.3 eV \cite{Srikant.1998}. Es eignet
sich also wegen der Wellenlänge \mbox{$\uplambda\approx$ 390 nm} für die
Herstellung besonders dünner Drähte, die trotzdem noch das Licht ihrer Lasermode
führen können. Einer der großen Vorteile liegt in der Herstellung solcher
Strukturen unter Ausnutzung von Selbstorganisationsmechanismen. Mithilfe des
Vapor-Liquid-Solid-Verfahrens (VLS) können unzählige dieser Nanostrukturen mit
verschiedensten Durchmessern und Längen sowie hoher Kristallinität gleichzeitig
gewachsen und damit technisch aufwendige Produktionsschritte wie
Lithographieverfahren einfach \mbox{umgangen} werden \cite{Roeder.Diss}.\\ Zur
Messung der Emissionseigenschaften wird in der ``Top-View''-Geometrie, einer
Sichtperspektive von oben auf den flach auf dem Substrat liegenden Nanodraht
herab, die Polarisation winkelaufgelöst analysiert und mittels Stokes-Parametern
charakteri- siert. Zu diesem Zweck wurden die Methode der rotierenden
$\uplambda$/4-Platte und eine Fourieroptik eingesetzt. Um weitere Erkenntnisse
über das Verhalten der Emission bei externen (Stör-)Feldern zu gewinnen, werden
die ZnO-Nanodrähte zudem einem Magnetfeld ausgesetzt. Dazu werden die Nanodrähte
auf ein magnetisierbares Substrat \mbox{aufgebracht}. Durch Ausnutzung von
``Proximity-Effekten'' soll eine spontane Gleichrichtung der Elektronen- und
Lochspins mit den Spins des \mbox{magnetisierten} Substrats stattfinden und so
eine Spinpolarisation induziert \cite{Epstein.2002}, deren Auswirkungen auf die
Emissionscharakeristik untersucht werden.\\ Hierzu wurde im Rahmen dieser
Masterthesis ein Fourieraufbau in den \mbox{bestehenden} Mikro-Photolumineszenz
($\upmu$-PL)-Aufbau integriert. Es wurde ein drehbares Helmholtz- spulenpaar
dimensioniert, das es erlaubt, die Magnetisierungsrichtung des Substrates
während der Messung \textit{in situ} zu verändern. Weiterhin wurde ein
Matlab-Auswertungs- skript zur Berechnung der Stokes-Parameter in großen Teilen
neu geschrieben, ver- bessert und auf die Messgeometrie zugeschnitten.
\chapter{Theoretische Grundlagen} \section{Das Materialsystem Zinkoxid (ZnO)}
\label{ZnOMat} Zinkoxid (ZnO) ist ein $\text{II}^\text{b}$-VI
Verbindungshalbleiter mit einer direkten Bandlücke von  $\text{E}_\text{g}=
(\text{3.37} \pm \text{0.01})$ eV bei Raumtemperatur am $\Upgamma$-Punkt
\cite{Klingshirn.2010}.
