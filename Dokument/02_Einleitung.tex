\chapter{Einleitung} Seit ungefähr 2016 erprobt die TK Softwareentwicklung in
agilen Teams. Das Team Online and Mobile Projects, in dem diese Arbeit entstand,
ist eines der Pilotprojekte dieses Vorstoßes. Das Team ist unter anderem mit der
Betreuung und Weiterentwicklung der TK-App betraut. Die TK-App versucht,
Verwaltungsprozesse, Kundeninteraktion im Allgemeinen, nutzerfreundlich
umzusetzen.\\ In Meine TK, der Web-Plattform für Versicherte der TK, sind einige
der von uns für die App geplanten Prozesse bereits vorhanden. Teil der agilen
Planung ist das \gls{mvp}. \glspl{mvp} sind minimal nutzbare Produkte. Im Rahmen
eines  \gls{mvp}s ist es denkbar, einen Prozess im ersten Schritt nur teilweise
in der App zu verwirklichen, den Nutzer ab einem bestimmten Punkt an Meine TK
weiterzuleiten und den Prozess dort fortzusetzen. Hierzu müssen Nutzer sich
derzeit stets erneut einloggen. Ziel dieser Arbeit ist, den Prozess der
Weiterleitung für Nutzer angenehmer zu gestalten, indem die erneute Eingabe
seiner Daten für ihn entfällt.\\ Die Weiterleitung von Zugriffsrechten ist seit
2012 mit der Veröffentlichung von \gls{OAuth}
\textit{grundsätzlich} keine schwere Aufgabe mehr. Da die TK jedoch hohe
Ansprüche an ihrer Sicherheitsstandards stellt und die Sicherheit der
Implementierung eines Protokolls nicht durch die Sicherheit des Protokolls
sichergestellt ist, wie~\cite{Sun.2012,Hu.2014,Yang.2016} zeigten, ist diese
Arbeit entstanden.\\ Die Frage, ob sich OAuth, das als Autorisierungsverfahren
gedacht ist, für die Authentifizierung eines Nutzers eignet, wird in dieser
Arbeit elegant umschifft.
