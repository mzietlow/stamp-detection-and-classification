\setacronymstyle{long-short-desc}

% ----------------------------- Acronyms -----------------------------
\newacronym[description={Organisation von
Workinggroups, die sich mit je einem spezifischen Thema befassen. Teil der
Internet Society (ISOC)}]{ietf}{IETF}{Internet Engineering Task Force}
\newacronym[description={hostet Nutzerinformationen}]{idp}{IdP}{Identity
Provider}
\newacronym[description={Umsetzung eines Features oder Produktes mit zwar
minimalem Funktionsumfang, aber dennoch konkretem Mehrwert für den Nutzer},
plural={MVPs}] {mvp}{MVP}{Minimal Viable Product}
\newacronym[description={Protokoll zur hybriden Verschlüsselung von
Datenübertragungen, Grundlage für HTTPS}]{TLS}{TLS}{Transport Layer Security}

% ----------------------------- Glossary Entries -----------------------------

\newglossaryentry{OAuth} {
    name={OAuth 2.0},
    description={Protokoll zur API-basierten Autorisierung}
}
\newglossaryentry{RO} {
    name={Resource Owner},
    description={Im \gls{OAuth}-Kontext, Besitzer einer oder mehrer Ressourcen, auf die eine andere Anwendung Zugriff erhalten möchte.}
}
\newglossaryentry{Grant Type} {
    name={Grant Type},
    plural={Grant Types},
    description={bestimmt, wie Clients Authorization Grants von Resource Ownern
    erhalten können. Die vier Grant Types\: Authorization Code, Implicit, Resource Owner Password Credentials sowie Client Credentials wurden in RFC 6749
    definiert; auch eigene Grant Types zu definieren ist möglich~\citevgl[S.~23ff.]{RFC6749}}.
}
