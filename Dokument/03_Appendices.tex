\section{Basic Concepts}
\blindtext[1]

\section{Algorithms of related work}
\blindtext[2]

\subsection{Concepts of Bounding Box Encoding}\label{append:Concepts of Bounding Box Encoding}
\todo{Consider explaining log space, stripping into positive space, encoding in width-height}
\(box = {x_{\text{min}}, x_{\text{max}}, y_{\text{min}}, y_{\text{max}}}\)
\(box = {anchor, \text{height}, width}\)
\blindtext[1]

\subsection{Intersection Over Union for Bounding Boxes}\label{sect:Intersect Over Union}
Let the \gls{bbox} be 
\begin{equation}
    \text{bbox}=\left\{\text{bbox}_\text{center}, \text{bbox}_{\text{height}}, \text{bbox}_\text{height}\right\}
\end{equation}
and let default box be 
\begin{equation}
    \text{dbox}=\left\{\text{dbox}_\text{center}, \text{dbox}_\text{height}, \text{dbox}_\text{height}\right\}.
\end{equation}

The \gls{iou} is defined as
\begin{equation}
    IoU=\frac{\text{bbox}\cup \text{dbox}}{\text{bbox}\cap \text{dbox}}
\end{equation}
where
\begin{equation}
    \text{bbox}\cup \text{dbox}=\left(\text{bbox}_{\text{height}}*\text{bbox}_{\text{width}}\right) - \left(\text{dbox}_{\text{height}}*\text{dbox}_{\text{width}}\right) - \left(\text{bbox}\cap \text{dbox}\right)
\end{equation}
and
\begin{equation}
   \text{bbox}\cap \text{dbox} = \max\left\{\text{height}_{\text{bbox}, \text{dbox}}\right\}*\max\left\{\text{width}_{\text{bbox}, \text{dbox}}\right\}\\
\end{equation}
with
\begin{align}
    \max\left\{\text{height}_{\text{bbox}, \text{dbox}}\right\} &= \max\left\{\text{bbox}_{y_\text{max}}, \text{dbox}_{y_\text{max}}\right\}-\min\left\{\text{bbox}_{y_\text{min}}, \text{dbox}_{y_\text{min}}\right\}\\
    \max\left\{\text{width}_{\text{bbox}, \text{dbox}}\right\} &= \max\left\{\text{bbox}_{x_\text{max}}, \text{dbox}_{x_\text{max}}\right\}-\min\left\{\text{bbox}_{x_\text{min}}, \text{dbox}_{x_\text{min}}\right\}
\end{align}

\subsection{K-Means Clustering}\label{append:K-Means Clustering}
~\cite[386-390]{James.2017}
\subsection{Bounding Box Offsets}
\blindtext[1]

\subsection{Preprocessing}
\blindtext[3]

\subsection{Concepts of Neural Network Architectures}
\subsubsection{Convolutional Layer}\label{append:Convolutional Layer}
https://cs231n.github.io/convolutional-networks/
receptive field~\cite[335-345]{Goodfellow.2016}
\todo{mentioned that the convolutional operation preserves spatial structure}
\blindtext[1]
\blindtext[3]

\subsection{Loss Functions}
\blindtext[2]

\subsection{VGG16 and ResNet (Base Networks)}
\blindtext[3]


%\section{General Todos}
\todo{Keine absoluten Aussagen.}
\todo{Deckblatt anpassen}
\todo{Arab. Nummerierung im Appendix, Backmatter?}
\todo{Double check spatial frequency paper [8, 9]}
\todo{Glossar füllen, PCA etc.}
\todo{extract best practices from dissertation}
\todo{Cite FAST and ORB, Gaussian, mean shift, PCA, K-NN, decision tree, SVM, 
VGG16, FCN, deconvolution, VOC2011?}
\todo{PCA is in math environment\dots}
\todo{Generate a `class prototype' for stamp and no-stamp classes. (what is the
perfect stamp picture)}
\todo{Write a colour shifter for preprocessing, s.t.\ every colour is trained
and maybe even consider a color gradient shifter}
\todo{Why combine localization and classification, why not split both? Is it speed only?}
\todo{Add groundtruth to gls}
