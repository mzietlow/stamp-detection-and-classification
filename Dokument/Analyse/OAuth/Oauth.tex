\chapter{OAuth 2.0} \gls{OAuth} ist ein von der \gls{ietf} konzipierte Protokoll
zur Delegation von Zugriffsrechten für Anwendungssoftware. Zugriffsrechte werden
hierbei durch von dem Besitzer der entsprechenden Zielressource autorisierte,
zeitlich limitierte Access-Tokens abgebildet. Access-Tokens sind bereits häufig
über den Vergleich mit Konzerttickets erläutert worden. Wer im Besitz des
zeitlich beschränkten Tickets ist, in diesem Fall der Zeichenfolge des Tokens,
ist damit auch im Besitz der mit diesem einhergehenden Zugriffsrechte, e.g.\ dem
Zugriff auf eine fiktive UserInfo-API, die den Namen und die Adresse des über
den Token zuordenbar Nutzers zurückgibt. Access-Tokens sind also eine
Möglichkeit für den Ressourcenbesitzer, anderen Parteien limitierten Zugriff auf
Daten zu ermöglichen, die für ihn bereits von einer Anwendung verwaltet werden,
wobei das Zugriffsrecht unabhängig von der Identität der Partei allein auf den
Besitz des Tokens zurückgeht. Auf Access-Tokens soll in Abschnitt
~\ref{accessToken} näher eingegangen werden. Die Übergabe des Access-Tokens ist
durch sogenannte Flows organisiert, deren wichtigste, Authorization Grant Flow
sowie Implicit Grant, in Abschnitt~\ref{Flows} erläutert werden sollen. Die
Kommunikation innerhalb der Flows erfolgt mittels standardisierter Nachrichten,
die als JSON per HTTP-Methode ausgetauscht werden. Im Folgenden sollen die in
\gls{OAuth} beteiligten Parteien, Objekte, Nachrichtenformate und Flows
erläutert und zusammengeführt werden. Ferner soll \gls{OAuth} als
Authentifizierungsverfahren betrachtet und auf mögliche Probleme untersucht
werden, wofür auch die innerhalb der TK bereits vorhandene Umsetzung
betrachtet wird.

\section{Parteien}
Ressource Owner {Credentials},
User Agent {Browser},
Client Applications {needs access}
OAuthorization Server {Tokens, Tokens and Credential-Validation}
    \blindtext{}

\section{OAuth Pseudo Authentification}

\blindtext{}

\section{Flows}\label{Flows} Flows beschreiben die Interaktionen der vier
Parteien, die notwendig sind, um Zugriff auf eine geschützte Ressource zu
erhalten. Abbildung \ref{fig: Protocol Flow} stellt den grundlegenden
Kommunikationsablauf zwischen dem Client auf der einen sowie Resource Owner,
Authorization Server und Resource Server auf der anderen Seite dar. Der Protocol
Flow lässt sich auf drei grundlegende Schritte reduzieren.

\begin{labeling}{OAuthParteien}
    \item [Authorization Grant] der Client stößt einen der vier spezifizierten \glspl{Grant Type} an
    \item [Access Token Retrieval] vorausgesetzt, der Grant war erfolgreich, erhält der Client einen Access Token
    \item [Ressource Access] durch Vorlage des Access Tokens erhält der CLient Zugriff auf die entsprechende Ressource.
\end{labeling}


Im Folgenden soll insbesondere der~\nameref{ssec:authcode} sowie der
\nameref{ssec:implicit} erläutert werden. \nameref{ssec:passwordcred} und
\nameref{ssec:clientcred} sind Reduktionen der beiden vorgestellten Flows und
werden nicht behandelt.

\begin{figure}[h]
    \lstinputlisting[nolol]{Dokument/Analyse/OAuth/Flows/protocol-flow.ascii}
    \caption{Protocol Flow}\label{fig: Protocol Flow}
\end{figure}

\subsection{Authorization Code Grant}\label{ssec:authcode}
Der Authorization Code Grant ist der umfangreichste, aufgrund der vielen
Authentifizierungszwischenschritte aber auch der sicherste der in \gls{OAuth}
genutzten \glspl{Grant Type}. Vorausgesetzt, die Übermittlung der Daten erfolgt
per \gls{TLS}, ist die Zuverlässigkeit des Authorization Code Grant formal gesichert.

\begin{figure}[h]
    \lstinputlisting[nolol]{Dokument/Analyse/OAuth/Flows/authorization-code-grant.ascii}
    \caption{Protocol Flow}\label{fig: Protocol Flow}
\end{figure}

~\cite{Chari.2011}
\subsection{Implicit Grant}\label{ssec:implicit}
\subsection{Resource Owner Password Credentials Grant}\label{ssec:passwordcred}
\subsection{Client Credentials Grant}\label{ssec:clientcred}


\blindtext{}

\section{Access-Token}\label{accessTokens}
Bei Access-Tokens handelt es sich um serverseitig generierte, kryptographische
Werte~\citevgl[S.~273]{Rohr.2018}, also zufällige Zeichenfolgen, die nur schwer
erraten werden können~\citevgl[S.~72]{Petrlic.2016} Einer der zentralen
Kritikpunkte an \gls{OAuth} ist die Ungebundenheit der in RFC 6750
spezifizierten Bearer-Tokens, wie sie bereits zu Anfang dieses Kapitels
beschrieben wurden~\citevgl{Hammer.2012}. Eine Alternative zu Bearer-Tokens
stellen MAC-Tokens dar.

\subsubsection{MAC-Token}\label{ssec: MAC Token}
MAC Tokens, spezifiziert in dem \gls{ietf} HTTP MAC Draft, verhalten sich, um
die bereits genutzte Geld-Metapher wieder aufzugreifen, wie
Kreditkarten~\citevgl[S.~134]{Siriwardena.2014}. Hierzu wird dem Client ein
symmetrischer Schlüssel übermittelt, mit dem er den MAC berechnet. Da MAC-Tokens
in Kapitel~\ref{Implementierung} ausgeschlossen werden, sollen sie hier nicht
weiter betrachtet werden. Zusätzliche Informationen bezüglich der Berechnung
eines MAC-Tokens finden sich in~\cite[S.~136ff.]{Siriwardena.2014} sowie in
RFC 2104.

\section{OAuth Pseudo Authentification}
\section{Sicherheitsanalyse}
\subsection{Bearer-Tokens}
\subsection{JWT}
\subsection{SAML}


\blindtext{}

\chapter{Fazit}
Diese Arbeit gibt eine Übersicht, über die zwei grundlegenden OAuth Grant Types,
Authorization Code Grant und Implicit Grant. Es wurde diskutiert, wie OAuth
trotz unsichere User-Agents implementiert werden kann. Durch Betrachtung
beispielhafter Flow-Diagramme wurde die Übermittlung von Tokens im Klartext
ausgeschlossen. Aufgrund der besonderen Situation, dass die Websession zu Beginn
nicht auf den Nutzer bezogen werden kann, wurden auch MAC-Tokens verworfen. Die
Übermittlung wurde schließlich durch asymmetrisch verschlüsselte JWE Container
gelöst. Von der Übermittlung eines unverschlüsselten Access-Keys ist in jedem
Fall abzuraten. Eine Kopplung zwischen App- und Web-Client, wie in
Kapitel~\ref{ch:GekoppelteFlows} vorgeschlagen, ist zu empfehlen, da sie das
Risiko, das von entwendeten Parametern ausgeht, minimiert. Eine Verbesserung
würde die Erweiterung des vorgeschlagenen Flows (Abbildung~\ref{ls: Extended
Implicit Authorization TK}) auf einen Authorization Code Grant bieten. Aufgrund
der erhöhten Komplexität wurde davon in dieser Arbeit letztlich abgesehen.

