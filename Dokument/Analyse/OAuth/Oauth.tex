\chapter{OAuth 2.0} OAuth 2.0 ist ein von der \gls{ietf} konzipiertes und 2012 in den
RFC 6749 und RFC 6750 spezifiziertes Framework zur Delegation von
Zugriffsrechten für Software-Anwendungen. Die Zugriffsrechte auf eine Ressource
entspringen hierbei keinem Rollensystem, sondern werden durch den Eigentümer der
Ressource selbst autorisiert. So können Nutzer einer Anwendung limitierten
Zugriff auf Daten ermöglichen, die von einer anderen Anwendung verwaltet werden.
Die Kommunikation erfolgt mittels standardisierte Nachrichten, die per JSON und
HTTP ausgetauscht werden. Für den Austausch der Berechtigungen werden sogenannte
Flows genutzt. Im Folgenden sollen die in OAuth beteiligten Parteien, Objekte
und Flows erläutert und zusammengeführt werden. Ferner soll OAuth als
Authentifizierungsverfahren betrachtet und auf mögliche Probleme untersucht
werden, wofür auch die innerhalb der TK bereits vorhandene Umsetzung
herangezogen wird.

\section{Parteien}
\begin{labeling}{OAuthParteien}
    \item [Resource Owner] besitzt geschützte Ressourcen, die auf einem Resource Server gehostet werden.
    \item [Resource Server] hostet geschützte Ressourcen, nimmt Requests auf geschützte Ressourcen entgegen und gewährt Zugriff, sofern der übermittelte Access-Token valide ist.
    \item [Client Application] ist eine vom Resource Owner genutzte Anwendung, die Zugriff auf geschützte Ressourcen anfragt und entsprechende Requests an den Resource Server stellt.
    \item[Authorization Server] authentisiert den Resource Owner anhand seiner Zugangsdaten und stellt dem Client anschließend einen Access Token auf die angefragte Ressource zur Verfügung.
\end{labeling}

\section{Flows}
Protocol Flow
\subsection{Authorization Code Grant}
\subsection{Implicit Grant}
\subsection{Resource Owner Password Credentials Grant}
\subsection{Client Credentials Grant}


\blindtext{}

\section{OAuth Pseudo Authentification}
\section{Sicherheitsanalyse}
\subsection{Bearer-Tokens}
\subsection{JWT}
\subsection{SAML}


\blindtext{}

\section{Fazit}

