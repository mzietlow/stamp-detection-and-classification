\chapter{OAuth 2.0} \gls{OAuth} ist ein von der \gls{ietf} konzipiertes
Protokoll zur Delegation von Zugriffsrechten für Anwendungssoftware.
Zugriffsrechte werden hierbei durch, von dem Besitzer der entsprechenden
Zielressource autorisierte, zeitlich limitierte Access-Tokens abgebildet.
Access-Tokens sind bereits häufig über den Vergleich mit Konzerttickets oder
Bargeld erläutert worden. Wer im Besitz des zeitlich beschränkten Tickets ist,
der Zeichenfolge des Tokens, muss sich nicht weiter ausweisen und ist so auch im
Besitz der mit dem Token einhergehenden Zugriffsrechte, e.g.\ dem Zugriff auf
eine fiktive UserInfo-API, die den Namen und die Adresse des über den Token
zuordenbaren Nutzers zurückgibt. Access-Tokens sind also eine Möglichkeit für
den Ressourcenbesitzer, anderen Parteien limitierten Zugriff auf Daten zu
ermöglichen, die für ihn bereits von einer Anwendung verwaltet werden, wobei das
Zugriffsrecht, unabhängig von der Identität der Partei, allein auf den Besitz
des Tokens zurückgeht. Auf Access-Tokens soll in Abschnitt~\ref{accessToken}
näher eingegangen werden. Die Übergabe des Access-Tokens ist durch sogenannte
\glspl{Grant Type} strukturiert, deren wichtigste, Authorization Grant sowie
Implicit Grant, in Abschnitt~\ref{GrantTypes} erläutert werden sollen. Die
Kommunikation innerhalb der \glspl{Grant Type} erfolgt mittels standardisierter
Nachrichten, die als JSON per HTTP-Methode ausgetauscht werden. Im Folgenden
sollen die in \gls{OAuth} beteiligten Parteien, Objekte, Nachrichtenformate und
\glspl{Grant Type} erläutert und zusammengeführt werden. Ferner soll \gls{OAuth}
als Authentifizierungsverfahren betrachtet und auf mögliche Probleme untersucht
werden, wofür auch die innerhalb der TK bereits vorhandene Umsetzung betrachtet
wird.

\section{Parteien}
\begin{labeling}{OAuthParteien}
    \item [Resource Owner] besitzt geschützte Ressourcen, die auf einem Resource Server gehostet werden.
    \item [Resource Server] hostet geschützte Ressourcen, nimmt Requests auf geschützte Ressourcen entgegen und gewährt Zugriff, sofern der übermittelte Access-Token valide ist.
    \item [Client Application] ist eine vom Resource Owner genutzte Anwendung, die Zugriff auf geschützte Ressourcen anfragt und entsprechende Requests an den Resource Server stellt.
    \item[Authorization Server] authentisiert den Resource Owner anhand seiner Zugangsdaten und stellt dem Client anschließend einen Access Token auf die angefragte Ressource zur Verfügung.
\end{labeling}

\section{\glspl{Grant Type}}\label{GrantTypes} \glspl{Grant Type}, auch Flows
genannt, beschreiben die Interaktionen der vier Parteien untereinander, die
notwendig sind, um Zugriff auf eine geschützte Ressource zu erhalten.
Abbildung~\ref{fig: Protocol Flow} stellt den grundlegenden
Kommunikationsablaufzwischen dem Client auf der einen, sowie Resource Owner,
Authorization Server und Resource Server auf der anderen Seite dar. Jegliche
Netzwerkkommunikation zwischen den Parteien sollte hierbei unbedingt durch TLS
geschützt werden\citevgl[S.~99]{Siriwardena.2014}. Der Protocol Flow lässt sich
auf drei grundlegende Schritte reduzieren.

\begin{labeling}{Access-Token Retrieval}
    \item [Authorization Grant] der Client stößt einen der vier spezifizierten
    \glspl{Grant Type} an. Wichtig an Grants ist, dass Resource
    Owner-Credentials immer nur an den Authorization-Server übermittelt werden,
    nie an den Client.
    \item [Access-Token Retrieval] vorausgesetzt, der Grant war erfolgreich,
    erhält der Client einen Access-Token. In einigen Grant Types spiele
    Refresh-Tokens eine Rolle, in dieser Arbeit spielen sie keine\ldots
    \item [Resource Access] durch Vorlage des Access-Tokens erhält der Client
    Zugriff auf die entsprechende Ressource.
\end{labeling} Im Folgenden werden insbesondere der~\nameref{ssec:authcode}
sowie der \nameref{ssec:implicit} erläutert werden. Resource Owner Password
Credentials Grant und Client Credentials Grant sind Reduktionen der beiden
vorgestellten \glspl{Grant Type} und werden nicht weiter behandelt.

\begin{figure}[h]
    \scalebox{.5} {
        \lstinputlisting[nolol]{Dokument/Analyse/OAuth/Flows/protocol-flow.ascii}
    }
    \caption{Protocol Flow}\label{fig: Protocol Flow}
\end{figure} \noindent
\subsection{Authorization Code Grant}\label{ssec:authcode} Der Authorization
Code Grant ist immer dann möglich, wenn der Client auf einen Web-Browser
zugreifen kann und über ein eigenes Backend verfügt oder anderweitig dazu in der
Lage ist, sein Client Secret  geheimzuhalten. Zwar ist der Authorization Code
Grant, aufgrund der vielen Authentifizierungszwischenschritte der
umfangreichste, hierdurch aber auch der sicherste der in \gls{OAuth} genutzten
\glspl{Grant Type}. Vorausgesetzt, die Übermittlung der Daten erfolgt per
\gls{TLS}, ist die Zuverlässigkeit des Authorization Code Grant formal gesichert
\cite{Chari.2011}. Wie schon der~\nameref{fig: Protocol Flow}, lässt sich der
Authorization Code Grant leicht herunterbrechen. Er verfolgt zwei zentrale
Ziele: zum einen, den Erwerb eines Authorization Codes, zum anderen das Einlösen
dieses Authorization Codes gegen einen Access-Token.

\begin{figure}[h]
    \scalebox{.5}{
        \lstinputlisting[nolol]{Dokument/Analyse/OAuth/Flows/authorization-code-grant.ascii}
    }
    \caption{Authorization Code Grant}\label{fig: Authorization Code Grant}
\end{figure}

\subsubsection{\textit{Authorization Code Retrieval (1--6)}} Um Zugriff auf einen
Authorization Code zu erlangen, redirectet der Client den Resource Owner im
ersten Schritt an den Authorization Server. Aufgabe des Authorization
Servers ist es, die Identität des Clients sicherzustellen. In der Regel
geschieht dies über einen Login am Authorization Server oder eine bereits
bestehende Session. Ist die Identität des Resource Owners verifiziert, führt
wiederum der Authorization Server einen Redirect auf die vom Client
angegebene Redirect URI. Der Request des Redirects besteht aus zwei
erforderlichen Query Parameter:
\begin{labeling}{response}
    \item [response\_type] um einen Authorization Code zu erhalten,
    muss der Wert `code' gewählt werden.
    \item [client\_id] die Client ID wie in Kapitel~\ref{Parteien}
    beschrieben. Der erhaltene Authorization Code kann nur mit dem
    zugehörigen Client Secret eingelöst werden.
\end{labeling}
Weitere Parameter wie redirect\_uri, scope und state sind optional.
\begin{figure}[h]
    \scalebox{.8}{
        \lstinputlisting[xleftmargin=2.7cm]{Dokument/Analyse/OAuth/HTTPListings/AuthorizeRequest.ascii}
    }
    \caption{Authorization Request}\label{ls: Authorization Request}
\end{figure}
Vorausgesetzt, der Grant war erfolgreich, so wird der User Agent an die bei
der Client-Registrierung angegebene Redirect URI weitergeleitet, wobei der
Authorization Code im URI Fragment gesetzt ist. Wie der Client auf den
Parameter zugreift, ist nicht Teil der Spezifikation. Zu beachten ist, dass
der Authorization Code auf diesem Weg auch für den Nutzer und im Falle eines
manipulierten Browsers auch für etwaige Angreifer sichtbar ist. Um
Replay-Angriffen vorzubeugen, sollte jeder Authorization Code nur ein
einziges Mal genutzt werden können. Falls der Authorization Server eine
erneute Nutzung desselben Authorization Codes festellt, sollten alle bereits
ausgestellten Access-Tokens invalidiert werden.
\citevgl[S.~97]{Siriwardena.2014}
\begin{figure}[h]
    \scalebox{.8}{
        \lstinputlisting[xleftmargin=2.7cm]{Dokument/Analyse/OAuth/HTTPListings/AuthorizeResponse.ascii}
    }
    \caption{Authorization Response}\label{ls: Authorization Response}
\end{figure}

\subsubsection{\textit{Access-Token Retrieval (7--8)}}
Abschließend muss der Client den erhaltenen Authorization Code gegen
einen Access-Token eintauschen. Hierzu ist ein POST auf den Access-Token-
Endpoint des Authorization Servers auszuführen, der zwei erforderliche
Parameter entgegennimmt.

\begin{labeling}{grant}
    \item [grant\_type] um zwischen dem Authorization Code Grant und
    weiteren Grant Types wie dem Resource Owner Password Credentials Grant
    zu unterscheiden. Als Wert muss `authorization\_code' gewählt werden.
    \item [code] der im vorangegangenen Schritt erhaltene Authorization
    Code.
\end{labeling}
Der Access-Token Endpoint sollte mindestens per HTTP Basic Authentication
abgesichert werden. Client ID und Client Secret dienen als Zugangsdaten.
Da die Zugangsdaten in der Basic Authentication nicht verschlüsselt
übertragen werden, ist eine Übermittlung ausschließlich per HTTPS anzuraten.
Nur falls der Client nicht in der Lage sein sollte, HTTP Basic Authentication
auszuführen, dürfen Client ID und Client Secret im Body des Requests
übertragen werden \citevgl[S.~15f.]{RFC6749}.

\begin{figure}[h]
    \scalebox{.8}{
        \lstinputlisting[xleftmargin=2.7cm]{Dokument/Analyse/OAuth/HTTPListings/AccessTokenCurl.ascii}
    }
    \caption{Access-Token cURL}\label{ls: Access-Token cURL}
\end{figure}

Auf das vorangegangene cURL-Kommando sollte der Authorization Server mit
einem HTTP 200 OK und einem JSON-Body antworten. Die möglichen Token Types
werden in Kapitel~\ref{accessTokens} näher erläutert. Neben dem Access-Token
enthält das JSON Gültigkeitsdauer des Access-Tokens in Sekunden sowie
den weiter oben bereits kurz erwähnten Refresh-Token, der in dieser Arbeit
jedoch keine weitere Rolle spielt.

\begin{figure}[h]
    \scalebox{.8}{
        \lstinputlisting[xleftmargin=2.7cm]{Dokument/Analyse/OAuth/HTTPListings/AccessTokenCurlResponse.ascii}
    }
    \caption{Access-Token Response}\label{ls: Access-Token Response}
\end{figure}

\subsection{Implicit Grant}\label{ssec:implicit}
Anders als der \nameref{ssec:authcode} richtet sich der Implicit Grant an
unsichere Clients, die über kein eigenes Backend verfügen und auch anderweitig
nicht dazu in der Lage sind, ihr Client Secret zu schützen. Da Authorization
Codes nur mittels Client Secret in Access Codes eingetauscht werden können,
entfällt dieser Mechanismus für den Implicit Grant.

\begin{figure}[h]
    \scalebox{.5} {
        \lstinputlisting[nolol]{Dokument/Analyse/OAuth/Flows/implicit-grant.ascii}
    }
    \caption{Implicit Grant}\label{fig: Implicit Grant}
\end{figure} \noindent
Der Request an den Authorization Server ist in seinem Aufbau ident zu Listing~\ref{ls: Authorization Request}. Der Wert des response\_type ist auf `token'
anzupassen.

\begin{figure}[h]
    \scalebox{.8}{
        \lstinputlisting[xleftmargin=2.7cm]{Dokument/Analyse/OAuth/HTTPListings/ImplicitAuthorizeRequest.ascii}
    }
    \caption{Implicit Authorization Request}\label{ls: Implicit Authorization Request}
\end{figure} \noindent
Nach erfolgreicher Authentifizierung des Resource Owners in Schritt 1--4
antwortet der Authentication Server direkt mit einem Access-Token. Die Schritte
7--8 des \nameref{fig: Authorization Code Grant}s entfallen also.
Refresh-Token, die in dieser Arbeit ohnehin keine weitere Rolle spielen, werden
in Reaktion auf die geminderte Vertraulichkeit des Implicit Grants nicht mehr
bereitgestellt.

\begin{figure}[h]
    \scalebox{.8}{
        \lstinputlisting[xleftmargin=2.7cm]{Dokument/Analyse/OAuth/HTTPListings/ImplicitAuthorizeResponse.ascii}
    }
    \caption{Implicit Authorization Response}\label{ls: Implicit Authorization Response}
\end{figure}

\section{Access Token}\label{accessTokens}
Einer der zentralen Kritikpunkte an \gls{OAuth} ist die Ungebundenheit der in
RFC 6750 spezifizierten Bearer-Tokens. Bearer-Access-Tokens funktionieren, wie
zu Beginn dieses Kapitels bereits erläutert. Sie sind vollkommen unabhängig von
dem Client, der sie angefordert hat. Wer sich im Besitz eines Bearer-Tokens
befindet, der hat auch Zugriff auf die entsprechend geschützten Ressourcen.

\subsection{MAC Token}\label{ssec: MAC Token}
MAC Tokens, spezifiziert in dem \gls{ietf} HTTP MAC Draft, verhalten sich, um
die bereits genutzte Geld-Metapher wieder aufzugreifen, wie Kreditkarten.
~\citevgl[S.~134]{Siriwardena.2014} Hierzu wird dem Client ein symmetrischer
Schlüssel übermittelt, mit dem er den MAC berechnet. \todo{Erklären, weshalb ein
MAC mein Problem nicht löst. Der symmetrische Key ist genau so anfällig wie alles
andere. Authorization Code, Access Token. Wird halt unverschlüsselt versendet.}
Weitere Informationen bezüglich der Berechnung des MACs finden sich in
~\cite[S.~136ff.]{Siriwardena.2014} sowie in RFC 2104.

\section{OAuth Pseudo Authentification}
\section{Sicherheitsanalyse}
\subsection{Bearer-Tokens}
\subsection{JWT}
\subsection{SAML}


\blindtext{}

