\chapter{OAuth 2.0} \gls{OAuth} ist ein von der \gls{ietf} konzipierte Protokoll
zur Delegation von Zugriffsrechten für Anwendungssoftware. Zugriffsrechte werden
hierbei durch von dem Besitzer der entsprechenden Zielressource autorisierte,
zeitlich limitierte Access-Tokens abgebildet. Access-Tokens sind bereits häufig
über den Vergleich mit Konzerttickets erläutert worden. Wer im Besitz des
zeitlich beschränkten Tickets ist, in diesem Fall der Zeichenfolge des Tokens,
ist damit auch im Besitz der mit diesem einhergehenden Zugriffsrechte, e.g.\ dem
Zugriff auf eine fiktive UserInfo-API, die den Namen und die Adresse des über
den Token zuordenbar Nutzers zurückgibt. Access-Tokens sind also eine
Möglichkeit für den Ressourcenbesitzer, anderen Parteien limitierten Zugriff auf
Daten zu ermöglichen, die für ihn bereits von einer Anwendung verwaltet werden,
wobei das Zugriffsrecht unabhängig von der Identität der Partei allein auf den
Besitz des Tokens zurückgeht. Auf Access-Tokens soll in Abschnitt
~\ref{accessToken} näher eingegangen werden. Die Übergabe des Access-Tokens ist
durch sogenannte Flows organisiert, deren wichtigste, Authorization Grant Flow
sowie Implicit Grant, in Abschnitt~\ref{Flows} erläutert werden sollen. Die
Kommunikation innerhalb der Flows erfolgt mittels standardisierter Nachrichten,
die als JSON per HTTP-Methode ausgetauscht werden. Im Folgenden sollen die in
\gls{OAuth} beteiligten Parteien, Objekte, Nachrichtenformate und Flows
erläutert und zusammengeführt werden. Ferner soll \gls{OAuth} als
Authentifizierungsverfahren betrachtet und auf mögliche Probleme untersucht
werden, wofür auch die innerhalb der TK bereits vorhandene Umsetzung
betrachtet wird.

\section{Parteien}
\begin{labeling}{OAuthParteien}
    \item [Resource Owner] besitzt geschützte Ressourcen, die auf einem Resource Server gehostet werden.
    \item [Resource Server] hostet geschützte Ressourcen, nimmt Requests auf geschützte Ressourcen entgegen und gewährt Zugriff, sofern der übermittelte Access-Token valide ist.
    \item [Client Application] ist eine vom Resource Owner genutzte Anwendung, die Zugriff auf geschützte Ressourcen anfragt und entsprechende Requests an den Resource Server stellt.
    \item[Authorization Server] authentisiert den Resource Owner anhand seiner Zugangsdaten und stellt dem Client anschließend einen Access Token auf die angefragte Ressource zur Verfügung.
\end{labeling}

\section{Flows}
Protocol Flow
\subsection{Authorization Code Grant}
\subsection{Implicit Grant}
\subsection{Resource Owner Password Credentials Grant}
\subsection{Client Credentials Grant}


\blindtext{}

\section{Access Token}\label{accessTokens}
Einer der zentralen Kritikpunkte an \gls{OAuth} ist die Ungebundenheit der in
RFC 6750 spezifizierten Bearer-Tokens. Bearer-Access-Tokens funktionieren, wie
zu Beginn dieses Kapitels bereits erläutert. Sie sind vollkommen unabhängig von
dem Client, der sie angefordert hat. Wer sich im Besitz eines Bearer-Tokens
befindet, der hat auch Zugriff auf die entsprechend geschützten Ressourcen.

\subsection{MAC Token}\label{ssec: MAC Token}
MAC Tokens, spezifiziert in dem \gls{ietf} HTTP MAC Draft, verhalten sich, um
die bereits genutzte Geld-Metapher wieder aufzugreifen, wie Kreditkarten.
~\citevgl[S.~134]{Siriwardena.2014} Hierzu wird dem Client ein symmetrischer
Schlüssel übermittelt, mit dem er den MAC berechnet. \todo{Erklären, weshalb ein
MAC mein Problem nicht löst. Der symmetrische Key ist genau so anfällig wie alles
andere. Authorization Code, Access Token. Wird halt unverschlüsselt versendet.}
Weitere Informationen bezüglich der Berechnung des MACs finden sich in
~\cite[S.~136ff.]{Siriwardena.2014} sowie in RFC 2104.

\section{OAuth Pseudo Authentification}
\section{Sicherheitsanalyse}
\subsection{Bearer-Tokens}
\subsection{JWT}
\subsection{SAML}


\blindtext{}

\section{Fazit}

