\chapter{OAuth 2.0} OAuth 2.0 ist ein von der \gls{ietf} konzipiertes und 2012
in den RFC 6749 und RFC 6750 spezifiziertes Framework zur Delegation von
Zugriffsrechten für Software-Anwendungen. Die Zugriffsrechte auf eine Ressource
entspringen hierbei keinem Rollensystem, sondern werden durch den Eigentümer der
Ressource selbst autorisiert. So können Nutzer einer Anwendung limitierten
Zugriff auf Daten ermöglichen, die von einer anderen Anwendung verwaltet werden.
Die Kommunikation erfolgt mittels standardisierte Nachrichten, die per JSON und
HTTP ausgetauscht werden. Für den Austausch der Berechtigungen werden sogenannte
Flows genutzt. Im Folgenden sollen die in OAuth beteiligten Parteien, Objekte
und Flows erläutert und zusammengeführt werden. Ferner soll OAuth als
Authentifizierungsverfahren betrachtet und auf mögliche Probleme untersucht
werden, wofür auch die innerhalb der TK bereits vorhandene Umsetzung
herangezogen wird.

\section{Parteien}
Ressource Owner {Credentials},
User Agent {Browser},
Client Applications {needs access}
OAuthorization Server {Tokens, Tokens and Credential-Validation}
    \blindtext{}

\section{OAuth Pseudo Authentification}

\blindtext{}

\section{Flows}\label{Flows} Flows beschreiben die Interaktionen der vier
Parteien, die notwendig sind, um Zugriff auf eine geschützte Ressource zu
erhalten. Abbildung \ref{fig: Protocol Flow} stellt den grundlegenden
Kommunikationsablauf zwischen dem Client auf der einen sowie Resource Owner,
Authorization Server und Resource Server auf der anderen Seite dar. Der Protocol
Flow lässt sich auf drei grundlegende Schritte reduzieren.

\begin{labeling}{OAuthParteien}
    \item [Authorization Grant] der Client stößt einen der vier spezifizierten \glspl{Grant Type} an
    \item [Access Token Retrieval] vorausgesetzt, der Grant war erfolgreich, erhält der Client einen Access Token
    \item [Ressource Access] durch Vorlage des Access Tokens erhält der CLient Zugriff auf die entsprechende Ressource.
\end{labeling}


Im Folgenden soll insbesondere der~\nameref{ssec:authcode} sowie der
\nameref{ssec:implicit} erläutert werden. \nameref{ssec:passwordcred} und
\nameref{ssec:clientcred} sind Reduktionen der beiden vorgestellten Flows und
werden nicht behandelt.

\begin{figure}[h]
    \lstinputlisting[nolol]{Dokument/Analyse/OAuth/Flows/protocol-flow.ascii}
    \caption{Protocol Flow}\label{fig: Protocol Flow}
\end{figure}

\subsection{Authorization Code Grant}\label{ssec:authcode}
Der Authorization Code Grant ist der umfangreichste, aufgrund der vielen
Authentifizierungszwischenschritte aber auch der sicherste der in \gls{OAuth}
genutzten \glspl{Grant Type}. Vorausgesetzt, die Übermittlung der Daten erfolgt
per \gls{TLS}, ist die Zuverlässigkeit des Authorization Code Grant formal gesichert.

\begin{figure}[h]
    \lstinputlisting[nolol]{Dokument/Analyse/OAuth/Flows/authorization-code-grant.ascii}
    \caption{Protocol Flow}\label{fig: Protocol Flow}
\end{figure}

~\cite{Chari.2011}
\subsection{Implicit Grant}\label{ssec:implicit}
\subsection{Resource Owner Password Credentials Grant}\label{ssec:passwordcred}
\subsection{Client Credentials Grant}\label{ssec:clientcred}


\blindtext{}

\section{OAuth Pseudo Authentification}
\section{Sicherheitsanalyse}
\subsection{Bearer-Tokens}
\subsection{JWT}
\subsection{SAML}


\blindtext{}

\chapter{Fazit}
Diese Arbeit gibt eine Übersicht, über die zwei grundlegenden OAuth Grant Types,
Authorization Code Grant und Implicit Grant. Es wurde diskutiert, wie OAuth
trotz unsichere User-Agents implementiert werden kann. Durch Betrachtung
beispielhafter Flow-Diagramme wurde die Übermittlung von Tokens im Klartext
ausgeschlossen. Aufgrund der besonderen Situation, dass die Websession zu Beginn
nicht auf den Nutzer bezogen werden kann, wurden auch MAC-Tokens verworfen. Die
Übermittlung wurde schließlich durch asymmetrisch verschlüsselte JWE Container
gelöst. Von der Übermittlung eines unverschlüsselten Access-Keys ist in jedem
Fall abzuraten. Eine Kopplung zwischen App- und Web-Client, wie in
Kapitel~\ref{ch:GekoppelteFlows} vorgeschlagen, ist zu empfehlen, da sie das
Risiko, das von entwendeten Parametern ausgeht, minimiert. Eine Verbesserung
würde die Erweiterung des vorgeschlagenen Flows (Abbildung~\ref{ls: Extended
Implicit Authorization TK}) auf einen Authorization Code Grant bieten. Aufgrund
der erhöhten Komplexität wurde davon in dieser Arbeit letztlich abgesehen.

