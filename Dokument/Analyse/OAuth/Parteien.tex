\section{OAuth-Parteien}\label{Parteien}
\begin{labeling}{Authorization Server}
    \item [Resource Owner] besitzt geschützte Ressourcen, die auf einem Resource
    Server gehostet werden.
    \item [Resource Server] hostet geschützte Ressourcen, nimmt Requests auf
    geschützte Ressourcen entgegen und gewährt Zugriff, sofern der übermittelte
    Access-Token valide ist.
    \item [Client Application] ist eine vom Resource Owner genutzte Anwendung,
    die Zugriff auf geschützte Ressourcen anfragt und entsprechende Requests an
    den Resource Server stellt. Jeder Client muss sich am Authorization Server
    registrieren, woraufhin ihm eine Client-ID und ein Client-Secret zugewiesen
    werden.
    \item[Authorization Server] authentifiziert den Resource Owner anhand seiner
    Zugangsdaten und stellt dem Client je nach \gls{Grant Type} einen
    Authorization Code oder einen Access-Token auf die angefragte Ressource zur
    Verfügung. \citevgl[S.~6]{RFC6749}
\end{labeling}
