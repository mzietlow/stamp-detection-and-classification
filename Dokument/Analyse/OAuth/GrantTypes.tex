\section{\glspl{Grant Type}}\label{GrantTypes} \glspl{Grant Type}, auch Flows
genannt, beschreiben die Interaktionen der vier Parteien, die notwendig sind, um
Zugriff auf eine geschützte Ressource zu erhalten. Abbildung~\ref{fig: Protocol
Flow} stellt den grundlegenden Kommunikationsablauf zwischen dem Client auf der
einen sowie Resource Owner, Authorization Server und Resource Server auf der
anderen Seite dar. Der Protocol Flow lässt sich auf drei grundlegende Schritte
reduzieren.

\todo{AuthCode und Token Lifetimes ansprechen}

\begin{labeling}{OAuthParteien}
    \item [Authorization Grant] der Client stößt einen der vier spezifizierten
    \glspl{Grant Type} an
    \item [Access Token Retrieval] vorausgesetzt, der Grant war erfolgreich,
    erhält der Client einen Access Token
    \item [Ressource Access] durch Vorlage des Access Tokens erhält der Client
    Zugriff auf die entsprechende Ressource.
\end{labeling}


Im Folgenden soll insbesondere der~\nameref{ssec:authcode} sowie der
\nameref{ssec:implicit} erläutert werden. Resource Owner Password Credentials
Grant und Client Credentials Grant sind Reduktionen der beiden vorgestellten
\glspl{Grant Type} und werden nicht behandelt.

\begin{figure}[h]
    \scalebox{.6} {
        \lstinputlisting[nolol]{Dokument/Analyse/OAuth/Flows/protocol-flow.ascii}
    }
    \caption{Protocol Flow}\label{fig: Protocol Flow}
\end{figure} \todo{irgendwo
TLS betonen. Erwähnen, dass der Client nie die Credentials sieht}

\subsection{Authorization Code Grant}\label{ssec:authcode}
\todo{Refresh Token\ldots}
Der Authorization Code Grant ist immer dann möglich, wenn der Client auf einen
Web-Browser zugreifen kann und über ein eigenes Backend verfügt. Zwar ist der
Authorization Code Grant der umfangreichste, aufgrund der vielen
Authentifizierungszwischenschritte aber auch der sicherste der in \gls{OAuth}
genutzten \glspl{Grant Type}.Vorausgesetzt, die Übermittlung der Daten erfolgt
per \gls{TLS}, ist die Zuverlässigkeit des Authorization Code Grant formal
gesichert.\cite{Chari.2011} Wie schon der Protocol Flow,~\ref{fig: Protocol
Flow}, lässt sich der Authorization Code Grant leicht herunterbrechen. Er
verfolgt zwei zentrale Ziele: zum einen, den Erwerb eines Authorization Codes,
zum anderen das Einlösen dieses Authorization Codes gegen einen Access Token.

\begin{figure}[h]
    \scalebox{.6}{
        \lstinputlisting[nolol]{Dokument/Analyse/OAuth/Flows/authorization-code-grant.ascii}
    }
    \caption{Authorization Code Grant}\label{fig: Authorization Code Grant}
\end{figure}

\begin{labeling}{OAuthParteien}
    \item [Authorization Code Retrieval (Schritt 1--6)] Um Zugriff auf einen Authorization Code
    zu erlangen, redirectet der Client den Ressource Owner im ersten Schritt
    an den Authorization Server. Aufgabe des Authorization Servers ist es,
    die Identität des Clients sicherzustellen. In der Regel geschieht dies über
    einen Login am Authorization Server oder eine bereits bestehende Session.
    Ist die Identität des Resource Owners verifiziert, führt wiederum der
    Authorization Server einen Redirect auf die vom Client angegebene Redirect
    URI. Der Request des Redirects besteht aus zwei erforderlichen Query
    Parameter
    \begin{labeling}{RequestParams}
        \item [response\_type] um einen Authorization Code zu erhalten,
        muss der Wert `code' gewählt werden.
        \item [client\_id] die Client ID wie in Kapitel~\ref{Parteien}
        beschrieben. Der erhaltene Authorization Code kann nur mit dem
        zugehörigen Client Secret eingelöst werden.
    \end{labeling}
    Weitere Parameter wie redirect\_uri, scope und state sind optional.
    \begin{figure}[h]
        \lstinputlisting[xleftmargin=2.7cm]{Dokument/Analyse/OAuth/HTTPListings/AuthorizeRequest.ascii}
        \caption{Authorization Request}\label{ls: Authorization Request}
    \end{figure}
    Vorausgesetzt, der Grant war erfolgreich, so wird der User Agent an die bei
    der Client-Registrierung angegebene Redirect URI weitergeleitet, wobei der
    Authorization Code im URI Fragment oder als Query Parameter gesetzt ist. Wie
    der Client auf den Parameter zugreift, ist nicht Teil der Spezifikation. Zu
    beachten ist, dass der Authorization Code auf diesem Weg auch für den Nutzer
    und im Falle eines manipulierten Browsers auch für etwaige Angreifer
    sichtbar ist.
    \begin{figure}[h]
        \lstinputlisting[xleftmargin=2.7cm]{Dokument/Analyse/OAuth/HTTPListings/AuthorizeResponse.ascii}
        \caption{Authorization Response}\label{ls: Authorization Response}
    \end{figure}

    \item [Access Token Retrieval (7--8)]
    Abschließend muss der Client den erhaltenen Authorization Code gegen
    einen Access Token eintauschen. Hierzu ist ein POST auf den Access Token-
    Endpoint des Authorization Servers auszuführen, der zwei erforderliche
    Parameter entgegennimmt.
    \begin{labeling}{AccessTokenRequestParams}
        \item [grant\_type] um zwischen dem Authorization Code Grant und
        weiteren Grant Types wie dem Resource Owner Password Credentials Grant
        zu unterscheiden. Als Wert muss `authorization_code' gewählt werden.
        \item [code] der im vorangegangenen Schritt erhaltene Authorization
        Code.
    \end{labeling}
    Der Access Token Endpoint sollte mindestens per HTTP Basic Authentication
    abgesichert werden. Client ID und Client Secret dienen als Zugangsdaten.
    Da die Zugangsdaten in der Basic Authentication nicht verschlüsselt
    übertragen werden, ist eine Übermittlung ausschließlich per HTTPS anzuraten.
    Falls der Client nicht in der Lage sein sollte, HTTP Basic Authentication
    auszuführen, dürfen Client ID und Client Secret im Body des Requests
    übertragen werden.
\end{labeling}

\subsection{Implicit Grant}\label{ssec:implicit}
Der Implicit Grant

\blindtext{}
