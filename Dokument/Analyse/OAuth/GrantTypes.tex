\section{\glspl{Grant Type}}\label{GrantTypes} \glspl{Grant Type}, auch Flows
genannt, beschreiben die Interaktionen der vier Parteien untereinander, die
notwendig sind, um Zugriff auf eine geschützte Ressource zu erhalten.
Abbildung~\ref{fig: Protocol Flow} stellt den grundlegenden Kommunikationsablauf
zwischen dem Client auf der einen, sowie Resource Owner, Authorization Server
und Resource Server auf der anderen Seite dar. Jegliche Netzwerkkommunikation
zwischen den Parteien sollte hierbei unbedingt durch TLS geschützt
werden~\citevgl[S.~99]{Siriwardena.2014}. Der Protocol Flow (Abbildung~\ref{fig:
Protocol Flow}) lässt sich auf drei grundlegende Schritte reduzieren.

\begin{labeling}{Access-Token Retrieval}
    \item [Authorization Grant] der Client stößt einen der vier spezifizierten
    \glspl{Grant Type} an. Interessant an Grants ist, dass Resource
    Owner-Credentials in den meisten \glspl{Grant Type} nur an den
    Authorization Server übermittelt werden, nicht an den Client.
    \item [Access-Token Retrieval] vorausgesetzt, der Grant war erfolgreich,
    erhält der Client einen Access-Token. In einigen \glspl{Grant Type} spiele
    Refresh-Tokens eine Rolle, in dieser Arbeit spielen sie keine\ldots
    \item [Resource Access] durch Vorlage des Access-Tokens erhält der Client
    Zugriff auf die entsprechende Ressource.
\end{labeling} Im Folgenden werden insbesondere der~\nameref{ssec:authcode}
sowie der \nameref{ssec:implicit} erläutert. Resource Owner Password
Credentials Grant und Client Credentials Grant sind Reduktionen der beiden
vorgestellten \glspl{Grant Type} und werden nicht weiter behandelt.

\begin{figure}[h]
    \scalebox{.5} {
        \lstinputlisting[nolol]{Dokument/Analyse/OAuth/Flows/protocol-flow.ascii}
    }
    \caption{Protocol Flow~\protect\citevgl[S.~7]{RFC6749}}\label{fig: Protocol Flow}
\end{figure} \noindent
\subsection{Authorization Code Grant}\label{ssec:authcode} Der Authorization
Code Grant ist immer dann möglich, wenn der Client auf einen Web-Browser
zugreifen kann und über ein eigenes Backend verfügt oder anderweitig dazu in der
Lage ist, sein Client Secret geheimzuhalten~\citevgl[S.~24ff.]{RFC6749}. Zwar
ist der Authorization Code Grant, aufgrund der vielen
Authentifizierungszwischenschritte der umfangreichste, hierdurch aber auch der
sicherste der in \gls{OAuth} genutzten \glspl{Grant Type}. Vorausgesetzt, die
Übermittlung der Daten erfolgt per \gls{TLS}, ist die Zuverlässigkeit des
Authorization Code Grant formal gesichert~\cite{Chari.2011}. Wie schon
der~\nameref{fig: Protocol Flow}, lässt sich der Authorization Code Grant leicht
herunterbrechen. Er verfolgt zwei zentrale Ziele: zum einen, den Erwerb eines
Authorization Codes, zum anderen das Einlösen dieses Authorization Codes gegen
einen Access-Token. Abbildung~\ref{fig: Authorization Code Grant} zeigt den
Authorization Code Flow.

\begin{figure}[h]
    \scalebox{.5}{
        \lstinputlisting[nolol]{Dokument/Analyse/OAuth/Flows/authorization-code-grant.ascii}
    }
    \caption{Authorization Code
    Grant~\protect\citevgl[S.~24]{RFC6749}}\label{fig: Authorization Code Grant}
\end{figure}

\subsubsection{\textit{(1--6) Authorization Code Retrieval}} Um Zugriff auf einen
Authorization Code zu erlangen, redirectet der Client den Resource Owner im
ersten Schritt an den Authorization Server. Aufgabe des Authorization Servers
ist es, die Identität des Clients sicherzustellen. In der Regel geschieht dies
über einen Login am Authorization Server oder eine bereits bestehende Session,
die Authentifizierung ist grundsätzlich aber nicht Thema des Protokolls. Wie in
den meisten \glspl{Grant Type}, hat der Client selbst nie Zugriff auf die
Credentials des Resource Owners. Der Redirect besteht aus zwei erforderlichen Query
Parameter~\citevgl[S.~25ff.]{RFC6749}:
\begin{labeling}{response}
    \item [response\_type] um einen Authorization Code zu erhalten,
    muss der Wert `code' gewählt werden.
    \item [client\_id] die Client ID wie in Kapitel~\ref{Parteien}
    beschrieben. Der erhaltene Authorization Code kann nur mit dem
    zugehörigen Client Secret eingelöst werden.
\end{labeling}
Weitere Parameter wie redirect\_uri, scope und state sind optional. Abbildung
~\ref{ls: Authorization Request} zeigt einen beispielhaften Request.
\begin{figure}[h]
    \scalebox{.8}{
        \lstinputlisting[xleftmargin=2.7cm]{Dokument/Analyse/OAuth/HTTPListings/AuthorizeRequest.ascii}
    }
    \caption{Authorization Request~\protect\citevgl[S.~97]{Siriwardena.2014}}\label{ls: Authorization Request}
\end{figure}\noindent
Vorausgesetzt, der Grant war erfolgreich, wird der User Agent an die bei der
Client-Registrierung angegebene Redirect URI weitergeleitet, wobei der
Authorization Code im URI Fragment oder als Query Parameter gesetzt
ist~\citevgl[S.~25]{RFC6749} (Abbildung~\ref{ls: Authorization Response}). Wie
der Client auf den Parameter zugreift ist nicht Teil der Spezifikation. Zu
beachten ist, dass der Authorization Code auf diesem Weg auch für den Nutzer und
im Falle eines manipulierten Browsers auch für etwaige Angreifer sichtbar ist.
Um Replay-Angriffen vorzubeugen, sollte jeder Authorization Code nur ein
einziges Mal genutzt werden können~ \citevgl[S.~97]{Siriwardena.2014}. Falls der
Authorization Server eine erneute Nutzung desselben Authorization Codes
feststellt, sollten alle bereits ausgestellten Access-Tokens invalidiert werden.
\citevgl[S.~97]{Siriwardena.2014}
\begin{figure}[h]
    \scalebox{.8}{
        \lstinputlisting[xleftmargin=2.7cm]{Dokument/Analyse/OAuth/HTTPListings/AuthorizeResponse.ascii}
    }
    \caption{Authorization
    Response~\protect\citevgl[S.~98]{Siriwardena.2014}}\label{ls: Authorization Response}
\end{figure}

\subsubsection{\textit{(7--8) Access-Token Retrieval}} Abschließend muss der
Client den erhaltenen Authorization Code gegen einen Access-Token eintauschen.
Hierzu ist ein POST (Abbildung~\ref{ls: Access-Token cURL}) auf den
Access-Token-Endpoint des Authorization Servers auszuführen, der zwei
erforderliche Parameter entgegennimmt~\citevgl[S.~29]{RFC6749}:

\begin{labeling}{grant}
    \item [grant\_type] um zwischen dem Authorization Code Grant und
    weiteren Grant Types wie dem Resource Owner Password Credentials Grant
    zu unterscheiden. Als Wert muss `authorization\_code' gewählt werden.
    \item [code] der im vorangegangenen Schritt erhaltene Authorization
    Code.
\end{labeling}
Der Access-Token Endpoint sollte mindestens per HTTP Basic Authentication
abgesichert werden. Client ID und Client Secret dienen als Zugangsdaten. Da die
Zugangsdaten in der Basic Authentication nicht verschlüsselt übertragen werden,
ist eine Übermittlung ausschließlich per HTTPS
anzuraten~\citevgl[S.~97]{Siriwardena.2014} Nur falls der Client nicht in der
Lage sein sollte, HTTP Basic Authentication auszuführen, dürfen Client ID und
Client Secret im Body des Requests übertragen werden~\citevgl[S.~15f.]{RFC6749}.

\begin{figure}[h]
    \scalebox{.8}{
        \lstinputlisting[xleftmargin=2.7cm]{Dokument/Analyse/OAuth/HTTPListings/AccessTokenCurl.ascii}
    }
    \caption{Access-Token
    cURL~\protect\citevgl[S.~98]{Siriwardena.2014}}\label{ls: Access-Token cURL}
\end{figure} \noindent
Auf das vorangegangene cURL-Kommando in Abbildung~\ref{ls: Access-Token cURL}
sollte der Authorization Server mit einem HTTP 200 OK und einem JSON-Body
(Abbildung~\ref{ls: Authorization Response}) antworten~\citevgl[S.~31]{RFC6749}.
Die möglichen Token Types werden in Kapitel~\ref{accessTokens} näher erläutert.
Neben dem Access-Token enthält das JSON Informationen über die Gültigkeitsdauer
des Access-Tokens in Sekunden sowie den weiter oben bereits kurz erwähnten
Refresh-Token (ebd.), der in dieser Arbeit jedoch keine weitere Rolle spielt.

\begin{figure}[h]
    \scalebox{.8}{
        \lstinputlisting[xleftmargin=2.7cm]{Dokument/Analyse/OAuth/HTTPListings/AccessTokenCurlResponse.ascii}
    }
    \caption{Access-Token
    Response~\protect\citevgl[S.~98]{Siriwardena.2014}}\label{ls: Access-Token Response}
\end{figure}

\subsection{Implicit Grant}\label{ssec:implicit}
Anders als der \nameref{ssec:authcode} richtet sich der Implicit Grant
(Abbildung~\ref{fig: Implicit Grant} unsichere Clients, die über kein eigenes
Backend verfügen und auch anderweitig nicht dazu in der Lage sind, ihr Client
Secret zu schützen~\citevgl[S.~98]{Siriwardena.2014}. Da Authorization Codes nur
mittels Client Secret in Access Tokens eingetauscht werden können, entfallen
Authorization Codes für den Implicit Grant.
\begin{figure}[h]
    \scalebox{.5} {
        \lstinputlisting[nolol]{Dokument/Analyse/OAuth/Flows/implicit-grant.ascii}
    }
    \caption{Implicit Grant~\protect\citevgl[S.~32]{RFC6749}}\label{fig: Implicit Grant}
\end{figure} \noindent
Der Request an den Authorization Server (Abbildung~\ref{ls: Implicit
Authorization Request} ist in seinem Aufbau ident zu Abbildung~\ref{ls:
Authorization Request}. Der Wert des response\_type ist auf `token'
anzupassen~\citevgl[S.~19]{RFC6749}.
\begin{figure}[h]
    \scalebox{.8}{
        \lstinputlisting[xleftmargin=2.7cm]{Dokument/Analyse/OAuth/HTTPListings/ImplicitAuthorizeRequest.ascii}
    }
    \caption{Implicit Authorization
    Request~\protect\citevgl[S.~98]{Siriwardena.2014}}\label{ls: Implicit Authorization Request}
\end{figure} \noindent
Nach erfolgreicher Authentifizierung des Resource Owners in Schritt 1--4
antwortet der Authorization Server direkt mit einem Access-Token
(Abbildung~\ref{ls: Implicit Authorization Response}). Die Schritte 7--8 des
Authorization Code Grants entfallen also. Refresh-Tokens, die in dieser Arbeit
ohnehin keine weitere Rolle spielen, werden in Reaktion auf die geminderte
Vertraulichkeit des Implicit Grants nicht mehr bereitgestellt.
\begin{figure}[h]
    \scalebox{.8}{
        \lstinputlisting[xleftmargin=2.7cm]{Dokument/Analyse/OAuth/HTTPListings/ImplicitAuthorizeResponse.ascii}
    }
    \caption{Implicit Authorization
    Response~\protect\citevgl[S.~99]{Siriwardena.2014}}\label{ls: Implicit
    Authorization Response}
\end{figure}
