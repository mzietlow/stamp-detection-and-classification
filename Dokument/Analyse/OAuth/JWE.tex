\section{JWT und JWE}
Definiert in RFC 7159 ist JavaScript Object Notation (JSON) ein Format
zur Übertragung oder Speicherung von Daten, das sich in den vergangenen Jahren
als Alternative zu XML etabliert hat.~\citevgl[S.~201]{Siriwardena.2014}
JSON Web Token (JWT) ist ein JSON-basiertes Container-Format für
Datenübermittlung, das insbesondere als Dateiformat in OpenID Connect genutzt
wird. JWTs werden in der Praxis entweder signiert (JWS) oder verschlüsselt
(JWE).
Im Folgenden soll ein Beispielhafter JWE erstellt werden.

\subsection{JSON Web Encryption}
JWEs bestehen, je nach Anwendungsfall aus insgesamt fünf bis sechs
Attributen. JWEs mit sechs Attributen können sich an mehr als nur einen
Empfänger wenden.

\begin{figure}[h]
    \scalebox{.8}{
        \lstinputlisting[xleftmargin=2.7cm]{Dokument/Analyse/OAuth/JWEListings/jweComplete.ascii}
    }
    \caption{JWE}\label{ls: JWE}
\end{figure}

Listing~\ref{ls: JWE} zeigt ein fünfteiliges JWE mit den Attributen

\begin{labeling}{initialization\_vector}
    \item [header] enthält unverschlüsselte Metainformationen über die
    Verschlüsselung des Content Encryption Keys. Wichtige Elemente sind
    \begin{labeling}{alg}
        \item[alg] der Algorithmus zur Verschlüsselung des Content Encryption
        Keys
        \item[enc] der Algorithmus zur Verschlüsselung des Payloads, also der zu
        übermittelnden Daten.
    \end{labeling}
    \item [encrypted\_key] base64url-encodierter Content Encryption Key,
    verschlüsselt durch das in `alg' spezifizierten Algorithmus.
    \item [initialization\_vector] base64url-encodierter, zufällig generierter
    Wert, der im Rahmen mancher Ver- und Entschlüsselungsverfahren benötigt wird
    \item[ciphertext] base64url-encodiert, enthält die zu übermittelnden Daten der
    Nachricht. Verschlüsselt durch den Content Encryption Key nach dem in `enc'
    spezifizierten Algorithmus.
    \item[tag]
\end{labeling}

Zur Serialisierung werden die Attribute, falls sie es noch nicht sein sollten,
base64url-Encodiert und mit Punkten getrennt konkateniert.

\begin{figure}[h]
    \scalebox{.8}{
        \lstinputlisting[xleftmargin=2.7cm]{Dokument/Analyse/OAuth/JWEListings/jweEncoded.ascii}
    }
    \caption{Serialisiertes JWE}\label{ls: serializedJWE}
\end{figure}
\todo{Codebeispiel in den Anhang einfügen.}
