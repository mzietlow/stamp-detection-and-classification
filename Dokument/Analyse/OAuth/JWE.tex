\section{JWT und JWE}
Definiert in RFC 7159 ist JavaScript Object Notation (JSON) ein Format
zur Übertragung oder Speicherung von Daten, das sich in den vergangenen Jahren
als Alternative zu XML etabliert hat.~\citevgl[S.~201]{Siriwardena.2014}
JSON Web Token (JWT) ist ein JSON-basiertes Container-Format für
Datenübermittlung, das insbesondere als Dateiformat in OpenID Connect genutzt
wird. JWTs werden in der Praxis entweder signiert (JWS) oder verschlüsselt
(JWE).
Im Folgenden soll ein Beispielhafter JWE erstellt werden.

\subsection{JSON Web Encryption}
JWEs bestehen, je nach Serialisierungsverfahren aus insgesamt fünf bis sechs
Attributen, die Base64-Encodiert und mit Punkten getrennt konkateniert werden.

\begin{figure}[h]
    \scalebox{.8}{
        \lstinputlisting[xleftmargin=2.7cm]{Dokument/Analyse/OAuth/JWEListings/jweComplete.ascii}
    }
    \caption{JWE}\label{ls: JWE}
\end{figure}

Listing~\ref{ls: JWE} zeigt ein mit JWE Compact Serialization kompatibles,
fünfteiliges JWE mit den Attributen

\begin{labeling}{initialization\_vector}
    \item [header] enthält unverschlüsselte Metainformationen über die
    Verschlüsselung des Content Encryption Keys. Wichtige Elemente sind
    \begin{labeling}{alg}
        \item[alg] der Algorithmus zur Verschlüsselung des Content Encryption
        Keys
        \item[enc] der Algorithmus zur Verschlüsselung des Payloads, also der zu
        übermittelnden Daten.
    \end{labeling}
    \item [encrypted\_key] Base64-encodierter Content Encryption Key,
    verschlüsselt durch das in `alg' spezifizierten Algorithmus.
    \item [initialization\_vector] Base64-encodierter, zufällig generierter
    Wert, der im Rahmen mancher Ver- und Entschlüsselungsverfahren benötigt wird
    \item[ciphertext] Base64-encodiert, enthält die zu übermittelnden Daten der
    Nachricht. Verschlüsselt durch den Content Encryption Key nach dem in `enc'
    spezifizierten Algorithmus.
    \item[tag]
\end{labeling}

Header und Payload, wobei der Header
Informationen bezüglich der Verschlüsselung des Payloads enthält und der Payload
die zu übermittelnden Daten. Der Payload enthält

\begin{figure}[h]
    \scalebox{.8}{
        \lstinputlisting[xleftmargin=2.7cm]{Dokument/Analyse/OAuth/JWEListings/jweHeader.ascii}
    }
    \caption{JWE-Header Übersicht}\label{ls: JWEHeader}
\end{figure}
