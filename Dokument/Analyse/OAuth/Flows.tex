\section{Flows}\label{Flows} Flows beschreiben die Interaktionen der vier
Parteien, die notwendig sind, um Zugriff auf eine geschützte Ressource zu
erhalten. Abbildung \ref{fig: Protocol Flow} stellt den grundlegenden
Kommunikationsablauf zwischen dem Client auf der einen sowie Resource Owner,
Authorization Server und Resource Server auf der anderen Seite dar. Der Protocol
Flow lässt sich auf drei grundlegende Schritte reduzieren.

\begin{labeling}{OAuthParteien}
    \item [Authorization Grant] der Client stößt einen der vier spezifizierten \glspl{Grant Type} an
    \item [Access Token Retrieval] vorausgesetzt, der Grant war erfolgreich, erhält der Client einen Access Token
    \item [Ressource Access] durch Vorlage des Access Tokens erhält der CLient Zugriff auf die entsprechende Ressource.
\end{labeling}


Im Folgenden soll insbesondere der~\nameref{ssec:authcode} sowie der
\nameref{ssec:implicit} erläutert werden. \nameref{ssec:passwordcred} und
\nameref{ssec:clientcred} sind Reduktionen der beiden vorgestellten Flows und
werden nicht behandelt.

\begin{figure}[h]
    \lstinputlisting[nolol]{Dokument/Analyse/OAuth/Flows/protocol-flow.ascii}
    \caption{Protocol Flow}\label{fig: Protocol Flow}
\end{figure}

\subsection{Authorization Code Grant}\label{ssec:authcode}
Der Authorization Code Grant ist der umfangreichste, aufgrund der vielen
Authentifizierungszwischenschritte aber auch der sicherste der in \gls{OAuth}
genutzten \glspl{Grant Type}. Vorausgesetzt, die Übermittlung der Daten erfolgt
per \gls{TLS}, ist die Zuverlässigkeit des Authorization Code Grant formal gesichert.

\begin{figure}[h]
    \lstinputlisting[nolol]{Dokument/Analyse/OAuth/Flows/authorization-code-grant.ascii}
    \caption{Protocol Flow}\label{fig: Protocol Flow}
\end{figure}

~\cite{Chari.2011}
\subsection{Implicit Grant}\label{ssec:implicit}
\subsection{Resource Owner Password Credentials Grant}\label{ssec:passwordcred}
\subsection{Client Credentials Grant}\label{ssec:clientcred}


\blindtext{}
