\section{Access-Token}\label{accessTokens}
Bei Access-Tokens handelt es sich um serverseitig generierte, kryptographische
Werte~\citevgl[S.~273]{Rohr.2018}, also zufällige Zeichenfolgen, die nur schwer
erraten werden können~\citevgl[S.~72]{Petrlic.2016} Einer der zentralen
Kritikpunkte an \gls{OAuth} ist die Ungebundenheit der in RFC 6750
spezifizierten Bearer-Tokens, wie sie bereits zu Anfang dieses Kapitels
beschrieben wurden~\citevgl{Hammer.2012}. Eine Alternative zu Bearer-Tokens
stellen MAC-Tokens dar.

\subsubsection{MAC-Token}\label{ssec: MAC Token}
MAC Tokens, spezifiziert in dem \gls{ietf} HTTP MAC Draft, verhalten sich, um
die bereits genutzte Geld-Metapher wieder aufzugreifen, wie
Kreditkarten~\citevgl[S.~134]{Siriwardena.2014}. Hierzu wird dem Client ein
symmetrischer Schlüssel übermittelt, mit dem er den MAC berechnet. Da MAC-Tokens
in Kapitel~\ref{Implementierung} ausgeschlossen werden, sollen sie hier nicht
weiter betrachtet werden. Zusätzliche Informationen bezüglich der Berechnung
eines MAC-Tokens finden sich in~\cite[S.~136ff.]{Siriwardena.2014} sowie in
RFC 2104.
