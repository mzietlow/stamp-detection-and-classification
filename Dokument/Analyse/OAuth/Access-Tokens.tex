\section{Access Token}\label{accessTokens}
Einer der zentralen Kritikpunkte an \gls{OAuth} ist die Ungebundenheit der in
RFC 6750 spezifizierten Bearer-Tokens. Bearer-Access-Tokens funktionieren, wie
zu Beginn dieses Kapitels bereits erläutert. Sie sind vollkommen unabhängig von
dem Client, der sie angefordert hat. Wer sich im Besitz eines Bearer-Tokens
befindet, der hat auch Zugriff auf die entsprechend geschützten Ressourcen.

\subsection{MAC Token}\label{ssec: MAC Token}
MAC Tokens, spezifiziert in dem \gls{ietf} HTTP MAC Draft, verhalten sich, um
die bereits genutzte Geld-Metapher wieder aufzugreifen, wie Kreditkarten.
~\citevgl[S.~134]{Siriwardena.2014} Hierzu wird dem Client ein symmetrischer
Schlüssel übermittelt, mit dem er den MAC berechnet. \todo{Erklären, weshalb ein
MAC mein Problem nicht löst. Der symmetrische Key ist genau so anfällig wie alles
andere. Authorization Code, Access Token. Wird halt unverschlüsselt versendet.}
Weitere Informationen bezüglich der Berechnung des MACs finden sich in
~\cite[S.~136ff.]{Siriwardena.2014} sowie in RFC 2104.
