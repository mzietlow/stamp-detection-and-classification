\setacronymstyle{long-short}

% ----------------------------- Acronyms -----------------------------
% \glslongpluralkey and glsshortpluralkey for plural. might night work betimes
% deleting _all_ auxiliary files _might_ help.
\newacronym{ml}{ML}{Machine Learning}
\newacronym{skcm}{SKCM}{Simple K-Counting Machine}
\newacronym{ai}{AI}{Artificial Intelligence}
\newacronym{rnn}{RNN}{Recurrent Neural Network}
\newacronym{cnn}{CNN}{Convolutional Neural Network}
\newacronym{fcnn}{FCNN}{Fully Convolutional Neural Network}
\newacronym{yolo}{YOLO}{You Only Look Once}
\newacronym{ssd}{SSD}{Single Shot MultiBox Detector}
\newacronym{map}{mAP}{mean Average Precision}
\newacronym{svm}{SVM}{Support Vector Machine}
\newacronym{fcn}{FCN}{Fully Convolutional Neural Network}
\newacronym{iou}{IoU}{Intersection over Union}
\newacronym{ap}{AP}{Average Precision}
\newacronym{fp}{FP}{False Positive}
\newacronym{tp}{TP}{True Positive}
\newacronym{fn}{FN}{False Negative}
\newacronym{tn}{TN}{True Negative}
\newacronym{staver}{StaVer}{Stamp Verification Dataset~\cite{Micenkova.2015}}
\newacronym{pca}{PCA}{Principal Component Algorithm}
\newacronym[description={A common activation function for neural neworks.
\(f(x)=\max(0, x)\)}]{relu}{ReLU}{Rectified Linear Unit}
\newacronym[description={Umsetzung eines Features oder Produktes mit zwar
minimalem Funktionsumfang, aber dennoch konkretem Mehrwert für den Nutzer},
plural={MVPs}] {mvp}{MVP}{Minimal Viable Product}

% ----------------------------- Glossary Entries -----------------------------

\newglossaryentry{gt}{name={Ground Truth},
    text={ground truth},
    description={The correct labels for an example}
}

\newglossaryentry{bbox}{
    name={Bounding Box},
    text={bounding box},
    description={The smallest rectangle that can be drawn around an object,
    s.t.\ all pixels from that object are contained},
    plural={bounding boxes}
}


\newglossaryentry{feature map}{
    name={Feature Map},
    text={feature map},
    description={Output of a convolutional layer after applying an activation function like \gls{relu}}
}

\newglossaryentry{coco}{
    name={COCO},
    description={Common Objects in Context (COCO). A well known dataset for
    object detection, segmentation and captioning}
}

\newglossaryentry{voc}{
    name={Pascal VOC},
    description={Pascal VOC is a well known dataset/challenge for object detection
    and classification}
}

\newglossaryentry{oi}{
    name={Open Images},
    description={Google Open Images is a well known dataset/challenge for
    object detection and classification}
}

\newglossaryentry{anchor}{
    name={Anchor},
    text={anchor},
    description={Center of a bounding box},
}

\newglossaryentry{convolutional layer}{
    name={Convolutional Layer},
    text={convolutional layer},
    description={Composition of the convolutional operation and nonlinearity, 
    partially referring to the `complex layer terminology' 
    in~\cite[341]{Goodfellow.2016}}
}

\newglossaryentry{dense layer}{
    name={Dense Layer},
    text={dense layer},
    description={Also: fully connected layer. Composition of the matrix operation and nonlinearity, partially referring to the `complex layer terminology' 
    in~\cite[341]{Goodfellow.2016}}
}

\newglossaryentry{deconvolutional layer}{
    name={Deconvolutional Layer},
    text={deconvolutional layer},
    description={Composition of the deconvolutional operation and nonlinearity,
    partially referring to the `complex layer terminology' in~\cite[341]{Goodfellow.2016}}
}
