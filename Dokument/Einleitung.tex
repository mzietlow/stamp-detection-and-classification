\chapter{Einleitung} In der heutigen Hochtechnologiegesellschaft besteht auch
weiterhin ein \mbox{ungebremstes} Interesse daran, immer schnellere, kleinere
und effizientere Prozessoren zu \mbox{entwickeln}. Bisher konnte die
Halbleiterindustrie diesem weitestgehend nachkommen, dennoch zeichnen sich
Grenzen dieser Miniaturisierung ab, die unüberwindbar scheinen. Grundlegende
physikalische Begebenheiten wie Quanteneffekte führen dazu, dass funktionale
Strukturen nicht weiter verkleinert werden können, ohne sie massiv zu
beeinträchtigen und damit unbrauchbar werden zu lassen~\cite{Moore.2017}.\\  Um
den technischen Fortschritt auch in Zukunft vorantreiben zu können, bedarf es,
ob der Unumgänglichkeit dieser Effekte, neuer Denkansätze. Einer dieser Ansätze
umfasst die Verwendung optischer Schaltkreise. Diese bieten den Vorteil eines
Informationstransportes in Lichtgeschwindigkeit, im Gegensatz zu der deutlich
langsameren \mbox{Geschwindigkeit} der Elektronen in konventionellen
Schaltkreisen. Auch die Schaltzeiten sind gegenüber den herkömmlichen
elektronischen Bauteilen stark reduziert~\cite{Simonite.2010,Johnson.2015}, denn
bei elektronischen Bauteilen ist die Driftgeschwindigkeit der Elektronen ein
limitierender Faktor – dieser entfällt bei optischen Bauteilen. Durch beide
\mbox{Begebenheiten}, der schnelleren Transportgeschwindigkeit und den kürzeren
Schaltzeiten, \mbox{erhöht} sich der Datentransfer gegenüber herkömmlichen
elektronischen Prozessoren drastisch. Die Integrierte Optik (IO) als Technik der
nächsten Generation steckt aber noch in den Kinderschuhen und es wird ausgiebig
an der Verwirklichung geforscht~\cite{Touch.2017}, in Erwartung, dass der Markt
in den nächsten Jahren stark wachsen wird~\cite{Credence.2017}.\\ Für diese neue
Art des Rechnens wird eine gerichtete kohärente Lichtquelle benötigt – ein
Laser. Halbleiternanodrähte bilden eine solche Klasse miniaturisierter Laser und
könnten in Zukunft diese Lücke füllen. Sie besitzen einen Durchmesser von
wenigen hundert Nanometern, bei einer Länge von wenigen Mikrometern. Aufgrund
eines \mbox{Brechungsindex} größer eins, können sie Licht einer Wellenlänge
führen, die größer ist als ihr eigener Durchmesser~\cite{Zimmler.2010}. Damit
setzen Nanodrähte eine untere Grenze für die gerichtete Laseremission,
gleichzeitig fungieren sie aufgrund ihrer Struktur als natürliche Resonatoren
~\cite{Eichhorn.2013}. Sie besitzen Schaltzeiten im Bereich weniger
Pikosekunden, drei Größenordnungen unter den Schaltzeiten moderner
Transistoren~\cite{Sidiropoulos.2014,Qiu.2017}. Damit erfüllen Nanodrähte die
Anforderungen, die die Gesellschaft an ein solches System stellt – sie sind
schnell, sie sind klein und sie sind effizient.
