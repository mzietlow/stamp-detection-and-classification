\chapter{Anhang}
\section{Anleitungen zur Justage der $\upmu$PL}
\label{Justage}
Da der Aufbau der $\upmu$PL im Rahmen dieser Masterthesis stark verändert wurde, soll im Folgenden die Justage beschrieben werden.
\begin{itemize}
\item Laser einschalten
\item Strahl mit \textbf{M1} durch Pinhole \textbf{P1} fokussieren, auf freien Strahlengang achten
\item Mit \textbf{M3} auf Target \textbf{T2} an Beamdump \textbf{D2} zentrieren
\item Prisma verdrehen (NUR HINTERE SCHRAUBEN, NICHT SEITLICH) um durch Pinhole \textbf{P2} zu kommen
\item Beamwalk: Spiegel \textbf{M3} und Prisma abwechselnd verdrehen: Mit Spiegel auf \textbf{P2} zielen, mit Prisma auf Target \textbf{T2}, danach \textbf{P2} schließen. Für die Optimierung der Ausgangsintensität empfiehlt sich  beim Zielen auf \textbf{P2} die Leistungsdiode zu benutzen.
\item Aufweitungslinse \textbf{L1} einsetzen und auf Target \textbf{T2} zielen
\item Hilfsspiegel \textbf{M7} einbauen, Spiegel \textbf{M5} hochklappen.
\item Beamsplitter \textbf{BS3} so einstellen, dass Strahl durch die Rückseite des Schraubtargets \textbf{T3} im Objektivhalter fällt.
\item Gegebenenfalls die Weißlichtausleuchtung mit Spiegel \textbf{M6} wiederherstellen, Laser danach mit \textbf{M5} zurück auf Target \textbf{T3} zentrieren 
\item Mit Hilfspiegel \textbf{M5} wieder durch die Vorderseite von \textbf{T3} auf Schraubtarget \textbf{T4} im Filterrad \textbf{W} zielen. 
\item Beamwalk: Mit Hilfsspiegel \textbf{M5} auf Target \textbf{T4} zielen, mit Beamsplitter \textbf{B3} auf die Vorderseite von Target \textbf{T2}, bis der Strahl mittig die beiden Targets trifft. Gegebenenfalls Strahl während des Beamwalks mit \textbf{M5} nachjustieren.
\item Klappspiegel \textbf{M8} hochklappen, Filterrad \textbf{W} auf leer drehen, Schraubtarget \textbf{T4} entfernen und auf Target \textbf{T5} fokussieren
\item \textbf{M8} herunterklappen und mit Spiegel \textbf{M9} auf Schraubtarget \textbf{T6} am Monochromatoreingang fokussieren
\item Schraubtarget \textbf{T3} entfernen und Tubus mit Spiegel in den Objektivhalter schrauben, diesen verdrehen, um auf \textbf{T6} zu zielen
\item Shutter schließen, passendes Objektiv einschrauben, Hilfsspiegel \textbf{M7} entfernen, Probe an der Stage montieren und Stage einsetzen
\item Weißlichtquelle anschalten, Proben grob fokussieren, dann auf Monitor scharf stellen (durch Entfernung Objektiv-\textbf{PS-$\upmu$})
\item Optische Dichte \textbf{OD} 5-6 einstellen, Shutter auf, um Laserspot auf Monitor zu sehen
\item Prisma und Stellschrauben der Aufweitungslinse \textbf{L1} verdrehen bis ein schöner elliptischer Spot zu sehen ist
\item 355 nm-Filter an Filterrad eindrehen
\end{itemize}
\begin{figure}[h]
\centering
\includegraphics[width=0.75\textwidth]{Bilder/Anhang/justage}
\caption[Justage $\upmu$-PL]{Aufbau der $\upmu$-PL mit ComponentLibrary\cite{lib}.}
\end{figure}
\label{JustImg}
\newpage
\section{Leistungsmessung}
\label{Leistmess}
Für Intensitätsvariationen über mehrere Größenordnungen ist es wichtig zu wissen, wie sich das optische System verhält, um ggf. Nichtlinearitäten zu berücksichtigen. Dies wurde im Folgenden analysiert.
\begin{figure}[b]
\centering
\includegraphics[width=1\textwidth]{Bilder/Anhang/Bereichsvergleich}
\caption[Vergleich Leistungsmessung]{Vergleich der Leistungsmessung a) der unterschiedlichen Messbereiche der Leistungsmessungsdiode am Probenort im Weißlichtfokus (P2) mit der Intensität hinter dem Filterrad (P1) b) am Probenort (P2) im und außerhalb des Weißlichtfokus sowie ohne Objektiv.}
\label{P1P2}
\end{figure}
Hierzu wurde die Intensität an Si-Power 2 (P2) über der Intensität des Lasers an Si-Power 1 (P1) (s.\autoref{JustImg}) aufgetragen. In \autoref{P1P2} a) wurde die Messdiode in den Weißlichtfokus des Objektivs gerückt, sodass der Laserspot ungefähr die Ausdehnung besitzt, mit der in den Experimenten gearbeitet wird. Hierbei wurden zwei Messbereiche (bis 280 nW und bis 2.8 $\upmu$m) der Diode verglichen. Es zeigt sich, dass beide Messbereiche bis 80 nW identische Ergebnisse erzielen, es aber ab 100 nW zu deutlichen Abweichungen zwischen den beiden Messbereichen kommt. Aus der Messung von \autoref{P1P2} b)) geht hervor, dass die Messdiode im Weißlichtfokus schnell absättigt. Die Ursache hierfür liegt in der Funktionsweise der Diode begründet. Sie ist für Dauerstrichlaser (engl. \textit{continous wave}, kurz \textit{cw}) ausgelegt. Der Messbereich bis 280 nW schneidet also die Intensitätsspitzen des gepulsten Nd:YAG-Lasers über 280 nW ab. Da im zeitlichen Mittel aber die Intensität unter den 280 nW liegt, erkennt die Software im automatischen Betrieb nicht, dass sie in den nächst höheren Bereich hätte wechseln müssen. Dies führt zu einer Sättigung. Es ist also ratsam die Intensität außerhalb des Weißlichtfokus zu messen. Zudem sollte für alle Intensitätsmessungen (auch an Si-Power 1) für Messungen größer 100 nW der Bereich bis 2.8 $\upmu$W verwendet werden. 
\section{Herleitung Korrektur der Abbildung}
\label{correct}
Projiziert man eine Kugelfläche auf einen ebenen Detektor, wie beim Fourierimaging der Fall, so kommt es für höhere Raumwinkel zu Verzerrungen sowohl in der Skalierung als auch in den Intensitätsdichten. Diese Fehler lassen sich durch eine Transformation der Koordinatensysteme beheben. Dies soll im Folgenden ausführlich hergeleitet werden.
\\
\\
\begin{figure}[h]
\centering
\includegraphics[width=0.3\textwidth]{Bilder/Anhang/arcsin}
\caption[Fourier-Korrektur]{Abbildung der Kugelwelle auf den CCD Schirm.}
\end{figure}
Entsprechend der Abbildung ergibt sich
\begin{equation}
sin(\theta)=\frac{R}{r} \Longleftrightarrow \theta = arcsin\left(\frac{R}{r}\right) \text{ ,}
\label{Korr1}
\end{equation} 
gleichsam gilt über die Definition der numerischen Apertur
\begin{equation}
NA=n(\lambda) \, sin(\theta_{max}) \text{ ,}
\end{equation}
mit der optischen Dichte n($\uplambda$) zwischen Objektiv und Probe. Es gilt also
\begin{equation}
\frac{NA}{n(\lambda)}=\frac{R_{max}}{r} \Longleftrightarrow r=\frac{n(\lambda)}{NA} \cdot R_{max} \text{ ,}
\end{equation} für den größtmöglichen Abstand der Abbildung zum Mittelpunkt des Detektors R$_\text{max}$. Setzt man dieses nun in \autoref{Korr1} ein, so ergibt sich
\begin{equation}
\theta=arcsin\left(\frac{NA}{n(\lambda)}\, \frac{R}{R_{max}}\right)\text{ .}
\label{Korr4}
\end{equation}
Die Überführung der Messung am Detektor in den Raumwinkel skaliert also mit dem Arkussinus. Im Folgenden werden $\frac{\text{NA}}{\text{n}(\uplambda)}=\text{a}$ und $\frac{\text{R}}{\text{R}_\text{max}}=\text{z}$ gesetzt, sodass gilt
\begin{equation}
\theta=arcsin\left(az\right) \text{ .}
\end{equation}
Möchte man nun die Intensitäten korrigieren, so wird schnell ersichtlich, dass sich die Fläche der Bestrahlung vergrößert, die Intensitäten für höhere Raumwinkel also vermindert sind. Geht man davon aus, dass die gesamte Strahlung auf den Detektor fällt, so bleibt die Gesamtintensität erhalten und zwar in allen Koordinatensystemen. Es muss gelten:
\begin{equation}
I_{ges}=\int \limits_{\Omega^{\ast}} \! \mathrm{d}x \, \mathrm{d}y \,\, \frac{I}{A}(x,y) = \int \limits_\Omega \! \mathrm{d}\theta \, \mathrm{d}\varphi \,\, K(\theta) \cdot \frac{I}{A}(\theta, \varphi) \text{ ,}
\end{equation}
mit einem Korrekturfaktor K($\uptheta$).\\
Der Korrekturfaktor folgt der Koordinatentransformation und ergibt sich zur Jacobi-Determinante
\begin{equation}
K(\theta)&= \mathrm{det} \left(\begin{matrix}
\frac{\partial x}{\partial \theta} & \frac{\partial x}{\partial \varphi} \\ 
\frac{\partial y}{\partial \theta} & \frac{\partial y}{\partial \varphi}
\end{matrix}  \right)=r^2 \cdot sin(\theta) \cdot cos(\theta)\\
\text{mit } x&= r \cdot cos(\theta) \cdot cos(\varphi) \notag\\
			y&= r \cdot cos(\theta) \cdot sin(\varphi) \notag \text{ ,}
\end{equation}
daraus folgt
\begin{equation}
I_{ges}=\int \limits_{\Omega^{\ast}} \! \mathrm{d}x \, \mathrm{d}y \,\, \frac{I}{A}(x,y) = \int \limits_\Omega \! \mathrm{d}\theta \, \mathrm{d}\varphi \,\, r^2 \cdot sin(\theta) \cdot cos(\theta)\, \frac{I}{A}(\theta, \varphi) \text{ ,}
\end{equation}
mit der Definition des Raumwinkels als
\begin{equation}
\mathrm{d}\omega= r \, \mathrm{d}\theta \,\,\, r \,sin(\theta) \,\mathrm{d}\varphi
\end{equation}
ergibt sich das Integral zu
\begin{equation}
I_{ges}=\int \limits_{\Omega^{\ast}} \! \mathrm{d}x \, \mathrm{d}y \,\, \frac{I}{A}(x,y) = \int \limits_\Omega \! \mathrm{d}\omega \, cos(\omega) \,\, \frac{I}{A}(\omega) \text{ ,}
\end{equation}
mit dem Korrekturfaktor
\begin{equation}
K(\omega)=cos (\omega) \text{ .}
\end{equation}
Jeder Pixel auf der CCD integriert nun die lokale Intensitätsdichte über seine infinitesimal kleine Fläche und fixiert somit örtlich einen Intensitätswert.
\begin{equation}
I_{ges}= \sum_N \int \limits_{Pixel(x,y) } \! \mathrm{d}x \, \mathrm{d}y \,\, \frac{I}{A}(x,y) = \sum_N \int \limits_{Pixel(\omega)} \! \mathrm{d}\omega \, cos(\omega) \, \frac{I}{A}(\omega)
\end{equation}
Alle Pixel im Raumwinkelsystem sind gleich groß. Für eine kleine Ausdehnung der Pixel im Raumwinkelsystem gegenüber der Krümmung der Raumwinkelkugel lassen sie sich als annähernd planar beschreiben.  
Für jeden einzelnen Pixel N mittelt sich also der Raumwinkel zu
\begin{equation}
\overline{\omega} = \frac{\omega_{begin} + \upomega_{end}}{2} \text{ .}
\end{equation}
Damit lässt sich das Integral zu
\begin{equation}
\sum_N \int \limits_{Pixel(x,y)} \! \mathrm{d}\omega \,\, cos(\overline{\omega}) \, \frac{I}{A}(\omega) = \sum_N  cos(\overline{\omega})  \cdot \int  \limits_{Pixel(\omega)} \! \mathrm{d}\omega  \, \frac{I}{A}(\omega) \text{ .}
\end{equation}
vereinfachen, da der Faktor $cos(\overline{\omega})$ vom Integral unabhängig wird und somit vorgezogen werden kann. Löst man nun beide Integrale, ergeben sich die aufintegrierten Intensitätsdichten der Pixel der jeweiligen Systeme. Es gilt also die Gleichung
\begin{equation}
\sum_N I_{Pixel}(x,y) = \sum_N cos(\overline{\omega}) \, I_{Pixel}(\overline{\omega})  \text{ ,}
\end{equation}
mit der sich die Intensitäten der Pixel der jeweiligen Systeme ineinander transformieren lassen. Für den einzelnen Pixel gilt also die Gleichung
\begin{equation}
I_{Pixel}(\overline{\omega}) = \frac{1}{cos(\overline{\omega})} \, I_{Pixel}(x,y) \text{ ,} 
\label{Korr2}
\end{equation}
die die Pixeln der CCD-Kamera in die der Raumwinkelpixel transformiert.\\
\\
Vereinfacht gesprochen muss also Folgendes berücksichtigt werden:\\ 
Denken wir uns in ein Bild in dem ein Strahl der \textbf{Breite$=$1 Pixel} und der \textbf{Intensität$=$1} gerade also im $\upomega=\textbf{0$^\circ$}$ \textbf{Winkel} auf die CCD-Kamera fällt. Der Strahl hat nun die Intensitätsdichte$=$1.\\
Der selbe Strahl trifft nun im $\upomega=\textbf{60$^\circ$}$ \textbf{Winkel} auf die CCD, er trifft  auf genau \textbf{zwei Pixel}. Jeder Pixel sieht also die \textbf{Intensität$=1/2$}.
Jeder Pixel stellt mathematisch einen Punkt dar. Der Abstand dieser Punkte ist gegeben durch den Abstand der Mittelpunkte ihrer Pixel. Für höhere Raumwinkel müssen also mehrere Pixel zu einem einzigen zusammengefasst werden. Dazu werde die  Abstände der mathematischen Punkte mit dem Korrekturfaktor K($\upomega$)=cos($\upomega$) verringert. Für die graphische Darstellung hat sich jetzt aber nur künstlich die Pixeldichte erhöht und auch wenn sich im mathematischen Sinne nun die Intensitätsdichte$=$ 2 x (Intensität$=1/2$) für die Raumwinkel korrigiert hat, bleibt die dargestellte Intensität der Einzelpixel unverändert zu gering, da die Pixel nicht aufsummiert werden, sondern jeder für sich stehen bleibt. Es reicht also nicht aus, die Abstände zu verändern, sondern jeder einzelne Punktintensität muss für die Darstellung korrigiert werden.
Mit der \autoref{Korr4} ergibt sich außerdem
\begin{equation}
I(\omega)= \frac{1}{\sqrt{1-a^2z^2}} \, I(R)
\label{Korr3}
\end{equation}
Folglich ergeben sich zwei Möglichkeiten die Messung anzupassen.
Entweder vor der Skalierung in die Raumwinkel über die Abstände R der Pixel zum Zentrum und der Korrektur \autoref{Korr3} oder nach der Skalierung in die Raumwinkel über den Winkel $\upomega$ nach der Korrektur \autoref{Korr2}.
\newpage
\section{Einmessung der Fourieroptik}
\label{fourier}
Nachdem die $\upmu$-PL fertig einjustiert wurde, kann begonnen werden, die Fourier- und die Abbildungslinse \textbf{L2} und \textbf{L4} einzumessen.
\begin{itemize}
\item Target \textbf{T4} in Filterwheel und \textbf{T3} in Objektivhalter einsetzen, Hilfsspiegel M7 einsetzen, Beamwalk nach obiger Anleitung
\item Abbildungslinse \textbf{L4} ausbauen, Schraubtarget \textbf{T6} einsetzen, mit Spiegel \textbf{M9} Strahl mittig aufs \textbf{T6} zentrieren
\item Fourierlinse \textbf{L2} in den Strahlengang klappen, Linse entfernen und Schraubtarget \textbf{T7} einsetzen, mit den Stellschrauben  das Target in den Laserstrahl zentrieren
\item Schraubtarget \textbf{T6} entfernen und Linse einsetzen 
\item Fourierlinse verdrehen, um auf Target \textbf{T6} zu zentrieren
\item Danach erneut Target \textbf{T7} einsetzen und erneut im Laserstrahl mittels Stellschrauben zentrieren, Target \textbf{T7} entfernen und \textbf{T8} in \textbf{L4} einsetzen
\item Mit den Stellschauben mittig in den Laser zentrieren, danach Target entfernen und Linse \textbf{L4} einsetzen
\item Verdrehen, bis mittig, dann erneut Target einsetzen und zentrieren
\item Shutter schließen, Abbildungs- und Fourierlinse einsetzen, Hilfsspiegel \textbf{M7} entfernen
\item  Mithilfe eines Spiegels hinter Pinhole \textbf{P3} Strahl aus dem üblichen Strahlengang auskoppeln und mithilfe eines zweiten Spiegels hinter dem Cage führen und mithilfe eines dritten Spiegels auf Target \textbf{T3} zentrieren, der Strahlengang muss nicht dem Beamwalk genügen.
\item \textbf{OD1} Filter auf ``Blank'' stellen, Objektiv einsetzen, um Fourierlinse mithilfe von Bodenplatten Führungsschiene bauen, Filter auf 4 Größenordnungen stellen
\item Mithilfe der CCD und dem Focusmodus der Software bei 5000 $\upmu$m Slitweite Fourierbild des Laserstrahls bei kurzer Belichtungszeit darstellen
\item Da Laserstrahl paralleles Licht, bildet Fourierlinse den Strahl in einem Punkt ab, durch Variation des Abstandes der Linse \textbf{L2} zum Objektiv komplette Intensität auf möglichst wenige Pixel konzentrieren. Wenn erfolgt, für Linse \textbf{L4} wiederholen.
\end{itemize}
\newpage
\section{Fehlerabschätzung und Hintergrundrauschen}
\label{Fehler}
Um die Aussagekraft der ermittelten Messwerte einordnen zu können, sollen im Folgenden mögliche Abweichungen und die daraus resultierenden Fehler berechnet werden. Sämtliche hier aufgeführte Formeln wurden in das Matlab-Auswertungskript implementiert und für jede Messung automatisiert berechnet. Es sollen im Folgenden die maßgeblichen Fehler betrachtet werden.
\subsection{Fehler der Fourierkoeffizienten}
\subsubsection{Schwankung der PL-Emission}
Die PL-Emission der Nanodrähte korreliert eng mit der Anregungsleistung des verwendeten Nd:YAG-Lasers. Diese unterliegt jedoch gewissen Schwankungen, die berücksichtigt werden müssen und sich direkt auf die Emission auswirken. Um die Effekte möglichst linear beschreiben zu können, muss der Nanodraht deutlich im Lasingbereich angeregt werden, da es ansonsten schnell durch die superlineare Steigung im Bereich der ASE zu hohen Fehlern kommen kann. Die mittlere Anregungsdichte p des Lasers wird mithilfe der Messdiode bestimmt und mit einer Standardabweichung s$_\text{p}$ ausgegeben. Es wurden jeweils N$=$500 Messintervalle à 0.2 s gemittelt. Für ein Vertrauensniveau von 2$\sigma$ ergibt sich mit $\uptau=\text{1.965}$ der Fehler zu
\begin{equation}
\Delta p = \frac{s_p}{\sqrt{N}} \cdot \tau \text{ .}
\end{equation}
Dieser Fehler wiederum setzt sich in der PL-Intensität fort. Anhand der Casperson-Anpassung, die die Anregungsdichte mit der PL-Intensität des Nanodrahtes in Beziehung setzt, kann dieser Fehler ermittelt werden.\\
Es gilt:
\begin{equation}
\Delta I_{PL}= I_{PL}(p+ \Delta p)-I_{PL}(p) \qquad \text{bzw.} \qquad \frac{\Delta I_{PL}}{p}= \frac{I_{PL}(p + \Delta p)-I_{PL}(p)}{I_{PL}} \text{ .}
\end{equation}
Für den Intensitätsfehler eines jeden Bildes gilt also:
\begin{equation}
\Delta I_{n} = \frac{\Delta I_{PL}}{I_{PL}}\, \cdot \, I_{n} \text{ .}
\end{equation}
Damit folgt der Fehler der Fourierkoeffizienten aus \autoref{FourKoeff}:
\begin{equation}
\Delta A_I&=\frac{2}{N} \, \sum_{n=1}^N \left| I_n \cdot \left( \frac{\Delta I_{n}}{I_{n}}\right)\right| \text{ ,} \,\quad\qquad\qquad\, \Delta B_I=\frac{4}{N} \, \sum_{n=1}^N \left| I_n \, sin(2\theta) \cdot \left( \frac{\Delta I_{n}}{I_{n}}\right) \right| \text{ ,} \notag \\
\Delta C_I&=\frac{4}{N} \, \sum_{n=1}^N \left| I_n \, cos(4\theta) \cdot \left( \frac{\Delta I_{n}}{I_{n}}\right)\right| \text{ ,} \qquad \Delta D_I=\frac{4}{N} \, \sum_{n=1}^N \left| I_n \, sin(4\theta) \cdot \left( \frac{\Delta I_{n}}{I_{n}}\right)\right| \text{ .}
\label{FehlerI}
\end{equation}
\subsubsection{Genauigkeit der Rotation}
Zur Berechnung der Fourierkoeffizienten muss die $\uplambda$/4-Platte rotiert werden. Die Einstellung der richtigen Winkel ist also auf den Ablesefehler begrenzt. Dieser soll als ein Viertel Skalenteil resp. $\Updelta \uptheta =\text{0.5}^\circ$ angenommen werden. Nach
\begin{equation}
&\Delta A_\theta= \frac{\partial A}{\partial \theta} \cdot \Delta \theta \qquad\qquad\text{ ,} \Delta B_\theta= \frac{\partial B}{\partial \theta}\cdot \Delta \theta\text{ ,} \notag \\
&\Delta C_\theta= \frac{\partial C}{\partial \theta}\cdot \Delta \theta\text{ ,} \qquad\qquad \Delta D_\theta= \frac{\partial D}{\partial \theta}\cdot \Delta \theta\text{ ,}
\end{equation}
und \autoref{FourKoeff} ergibt sich für den Fehler der Fourierkoeffizienten
\begin{equation}
&\Delta A_\theta= 0 \text{ ,} \qquad\qquad\qquad\qquad\qquad\qquad\quad\,\,\, \Delta B_\theta = \frac{4}{N} \, \sum_{n=1}^N \left| 2\, I_n \, cos(2\,\theta)\cdot \frac{\pi}{180^\circ} \, \Delta\theta \right| \text{ ,} \notag \\
&\Delta C_\theta=\frac{4}{N} \, \sum_{n=1}^N \left| 4\, I_n \, sin(4\,\theta)\cdot \frac{\pi}{180^\circ} \, \Delta\theta\right| \text{ ,} \quad\, \Delta D_\theta=\frac{4}{N} \, \sum_{n=1}^N \left| 4\, I_n \, cos(4\,\theta)\cdot \frac{\pi}{180^\circ} \, \Delta\theta\right| \text{ .}
\label{FehlerTheta}
\end{equation}\\
Zusammengefasst folgt aus \autoref{FehlerI} und \autoref{FehlerTheta} der Gesamtfehler nach:
\begin{equation}
&\Delta A_{ges}= \Delta A_I \text{ ,} \qquad\qquad\quad\, \Delta B_{ges}= \Delta B_I + \Delta B_\theta \text{ ,}\notag\\
&\Delta C_{ges}= \Delta C_I + \Delta C_\theta \text{ ,} \qquad \Delta D_{ges}= \Delta D_I + \Delta D_\theta \text{ .}
\end{equation} 
\subsection{Ausrichtung der optischen Elemente und Phasenverschiebung}
Bei der Ausrichtung der Transmissionsachse des Polarisators sowie der schnellen optischen Achse der $\uplambda$/4-Platte kommt es zu kleinen Abweichungen gegenüber dem gewünschten Koordinatensystem. Daraus resultieren kleine Verschiebungen der Stokes-Paramter untereinander.
Für den Winkel $\upalpha$ der $\uplambda$/4-Platte $\upalpha$ wurde ein Fehler von $\Updelta \upalpha=\text{1.2}^\circ$, für den Winkel des Polarisators ein Fehler von $\Updelta \upbeta=\text{1.8}^\circ$ ermittelt (s. \autoref{AusrOptEl}). Die Phasenverschiebung der $\uplambda$/4-Platte hängt stark vom Einfallswinkel des Lichtes zur Flächennormale der $\uplambda$/4-Platte ab. Der Fehler soll mit $\Updelta\updelta=3^\circ$ abgeschätzt sein. Durch die Position der Transmissionsachse wird ein Koordinatensystem für die Stokes-Parameter aufgespannt. Dieses System soll nun in das Koodinatensystem des Nanodrahtes transformiert werden, sodass die vertikale y-Achse der c-Achse des Nanodrahtes entspricht. Der Winkel zwischen Transmissionsachse und c-Achse des Nanodrahtes $\upbeta$ ist gegeben durch die Differenz beider Winkel $\Updelta \upbeta$ zwischen Transmissionsachse des Polarisators und Laborsystem sowie Nanodrahtachse und Laborsystem. Der Fehler des Winkels des Nanodrahts soll mit $\Updelta \upbeta_\text{NW}=\text{0.5}^\circ$ abgeschätzt werden. Somit wird der Gesamtfehler der Ausrichtung des Polarisators als $\Updelta \upbeta_{ges}=\text{2.3}^\circ$ angenommen.\\
Aus den \autoref{StokesGL5} folgen für die Stokes-Parameter die Fehlerfortpflanzungen:\\\\
\underline{für den S$_\text{0}$-Parameter}
\begin{equation}
\Delta S_0=& \Delta A_{ges} + \frac{1+cos(\delta)}{1-cos(\delta)} \Delta C_{ges} + \frac{\partial S_0}{\partial \delta} \cdot \Delta \delta \notag\\
=&\Delta A_{ges} +  \frac{1+cos(\delta)}{1-cos(\delta)} \, \Delta C_{ges} + 2\, \frac{sin(\delta)}{(1-cos(\delta))^2}\, \frac{\pi}{180^\circ}\, C \cdot \Delta \delta 
\end{equation}\\
\underline{für den S$_\text{1}$-Parameter}
\begin{equation}
\Delta S_1= \frac{2}{1-cos(\delta)}\,\left[ \Delta C_{ges}\cdot cos(2\,\beta) + \Delta D_{ges}\cdot sin(2\beta)\right] + \frac{\partial S_1}{\partial \delta} \cdot \Delta \delta + \frac{\partial S_1}{\partial \beta}\cdot \Delta \beta \notag
\end{equation}
\begin{multline}
\Delta S_1= \frac{2}{1-cos(\delta)}\,\left[ \left( \Delta C_{ges}+\frac{\pi}{180^\circ}\,\Delta \delta \, \frac{sin(\delta)}{1-cos(\delta)} \, C \right) \cdot cos(2\,\beta) \right]\\
+ \frac{2}{1-cos(\delta)}\,\left[ \left( \Delta D_{ges}+\frac{\pi}{180^\circ}\,\Delta \delta \, \frac{sin(\delta)}{1-cos(\delta)} \, D \right) \cdot cos(2\,\beta) \right]+ \frac{\partial S_1}{\partial \beta}\cdot \Delta \beta \notag
\end{multline}
\begin{multline}
\Delta S_1= \frac{2}{1-cos(\delta)}\,\left[ \left( \Delta C_{ges}+\frac{\pi}{180^\circ}\,\left( \Delta \delta \, \frac{sin(\delta)}{1-cos(\delta)} \, C + 2\,D \cdot \Delta\beta\right) \right) \cdot cos(2\,\beta) \right]\\
\qquad\,\,\, + \frac{2}{1-cos(\delta)}\,\left[ \left( \Delta D_{ges}+\frac{\pi}{180^\circ}\,\left( \Delta \delta \, \frac{sin(\delta)}{1-sin(\delta)} \, D -2\, C \cdot \Delta\beta\right)\right) \cdot sin(2\,\beta) \right]
\end{multline}
\underline{für den S$_\text{2}$-Parameter}
\begin{multline}
\Delta S_2= \frac{2}{1-cos(\delta)}\,\left[ \left( \Delta D_{ges}+\frac{\pi}{180^\circ}\,\left( \Delta \delta \, \frac{sin(\delta)}{1-cos(\delta)} \, D + 2\,C \cdot \Delta\beta\right) \right) \cdot cos(2\,\beta) \right]\\
\qquad\,\,\,\, + \frac{2}{1-cos(\delta)}\,\left[ \left( \Delta C_{ges}+\frac{\pi}{180^\circ}\,\left( \Delta \delta \, \frac{sin(\delta)}{1-cos(\delta)} \, C -2\, D \cdot\Delta\beta\right)\right) \cdot sin(2\,\beta) \right]
\end{multline}
\underline{für den S$_\text{3}$-Parameter}
\begin{equation}
\Delta S_3&= \frac{2}{sin(x)}\left [\Delta B_{ges} + 2\, \frac{1}{tan(\delta)} \frac{\pi}{180^\circ}\, \Delta \delta\cdot B\right] 
\end{equation}
\underline{für die lineare Polarisation}
\begin{equation}
\Delta P_{Lin}&= \frac{2}{1-cos(\delta)}\, \left[ \frac{sin(\delta)}{1-cos(\delta)} \, \sqrt{C^2+D^2} \, \frac{\pi}{180^\circ} \, \Delta \delta + \frac{\Delta C\cdot C + \Delta D \cdot D}{\sqrt{C^2+D^2}}  \right]\notag\\
&=\frac{1}{1-cos(\delta)}\, \left[ sin(\delta)\, \sqrt{S_1^2+S_2^2} \, \frac{\pi}{180^\circ} \, \Delta \delta + 2\, \frac{\Delta C\cdot C + \Delta D \cdot D}{\sqrt{C^2+D^2}}  \right]
\end{equation}
\underline{für die Gesamtpolarisation}
\begin{multline}
\Delta P_{ges}= \frac{\left[ \left( \frac{4\,sin(\delta)}{(1-cos(\delta))^3}(C^2+D^2) + \frac{1}{tan(\delta)\, sin^2(\delta)}\, B^2 \right)\frac{\pi}{180^\circ}\,\Delta\delta\right]}{\sqrt{S_1^2+S_2^2+S_3^3}}\\
+ \frac{\left[ \frac{4}{(1-cos(\delta))^2}\, (\Delta C\cdot C + \Delta D \cdot D) + \frac{1}{sin^2(\delta)}\,\Delta B \cdot B) \right]}{\sqrt{S_1^2+S_2^2+S_3^3}}\notag
\end{multline}
\begin{multline}
\Delta P_{ges}= \frac{\left[ \left( \frac{sin(\delta)}{1-cos(\delta)}(S_1^2+S_2^2) + \frac{1}{tan(\delta)}\, S_3^2 \right)\frac{\pi}{180^\circ}\,\Delta\delta\right]}{\sqrt{S_1^2+S_2^2+S_3^3}}\\
+ \frac{\left[ \frac{4}{(1-cos(\delta))^2}\, (\Delta C\cdot C + \Delta D \cdot D) + \frac{1}{sin^2(\delta)}\,\Delta B \cdot B \right]}{\sqrt{S_1^2+S_2^2+S_3^3}}
\end{multline}
\underline{für den horizontalen Feldanteil}
\begin{equation}
\Delta \langle E_x \rangle^2 =\frac{1}{2}\, \left[ \Delta A_{ges} + \Delta C_{ges} \right]
\end{equation}
\underline{für den vertikalen Feldanteil}
\begin{equation}
\Delta \langle E_y \rangle^2 =\frac{1}{2}\, \left[ \Delta A_{ges} + \frac{3+cos(\delta)}{(1-cos(\delta))^2} \cdot \Delta C_{ges} +\frac{4\, sin(\delta)}{(1-cos(\delta))^2} \, \frac{\pi}{180^\circ} \, C \cdot \Delta\delta \right]
\end{equation}
Für den Fehler der Ausrichtung der $\uplambda$/4-Platte bedeutet diese Transformation eine zusätzliche Addition des Fehlers der Ausrichtung des Polarisators auf den ursprünglichen Fehler. So ergibt sich $\Updelta \upalpha_{ges} = \text{1.8} + \text{1.2} =\text{3}^\circ$. Der Fehler für die Unsicherheit des Winkels $\uptheta$ ergänzt sich also um die Terme:
\begin{equation}
&\Delta A_\theta= 0 \text{ ,} \\
&\Delta B_\theta = \frac{4}{N} \, \sum_{n=1}^N  2\, I_n \, cos(2\,\theta)\cdot \frac{\pi}{180^\circ} \, \Delta \alpha_{ges}  \text{ ,}\\
&\Delta C_\theta=\frac{4}{N} \, \sum_{n=1}^N  4\, I_n \, sin(4\,\theta)\cdot \frac{\pi}{180^\circ} \, \Delta \alpha_{ges} \text{ ,}\\
&\Delta D_\theta=\frac{4}{N} \, \sum_{n=1}^N  4\, I_n \, cos(4\,\theta)\cdot \frac{\pi}{180^\circ} \,  \Delta \alpha_{ges} \text{ .}
\end{equation}
Die Fehler der Ausrichtung der $\uplambda$/4-Platte $\Updelta \upalpha$ und der Transmissionsachse des Polarisators $\Updelta \upbeta$ sowie der Fehler der Verschiebung der $\uplambda$/4-Platte sind systematischer Natur. Für die vergleichenden Messungen mit angelegten Magnetfeldern ändern sie sich nicht. Sie führen zu den jeweils gleichen Abweichungen und können für den Vergleich ausgeblendet werden. 
\subsection{Kosmische Strahlung und Hintergrundrauschen}
\label{Rauschen}
Gleichzeitig lässt sich über die Vielzahl der aufgenommenen Hintergründe gleicher Messmethode in dieser Masterthesis eine Varianz der gemessenen Hintegrundintensität jedes Pixels ermitteln. Das Rauschen beträgt im Mittel $\Updelta= \text{0.2}$ Counts/s. Unterhalb dieser Grenze ist eine Änderung nicht signifikant.\\
Um zu verhindern, dass durch Rauschen oder kosmische Strahlen das Ergebnis verzerrt wurde, werden Pixel mit einem Polarisationsgrad DOP $\geq$ 2 aus den Bildern entfernt, sie erscheinen weiß. 
\section{Ergänzungen zur Berechnung der Stokes-Parameter}
\label{StokesFalsch}
In der Literatur finden sich vereinzelt Berechnungen $\text{B} \sim \sum_{\text{n}=\text{1}}^\text{N} \text{I}_\text{n} \, \text{sin}(\text{2}\uptheta)$ \cite{Berry.1977,Anleitung}, von solch einer Berechnungen wird in dieser Masterthesis Abstand genommen:\\
Für gesamtheitlich linkszirkular polarisiertes Licht ($\text{S}_\text{0}=\text{1}$, $\text{S}_\text{3}=\text{-1})$ ergibt sich nach \autoref{StokesGleichung} die Gesamtintensität zu
\begin{equation}
I(\theta)= \frac{1}{2}\left[ S_0 + S_3 \, sin(2\, \theta)\right]=\frac{1}{2} \, (1 - sin(2 \, \theta)) \text{ .}
\end{equation}
Eine Faltung nach \cite{Goldstein.2003} mit einem Kosinus- statt Sinusterm zur Berechnung des Fourierkoeffizienten B führt zum Widerspruch, nur mit dem Sinus ist folglich die korrekte Berechnung des S$_\text{3}$-Parameters möglich.
\begin{equation}
S_3^\ast &=B^\ast= \frac{2}{\pi} \int_0^{2\,\pi} \, d\theta \, I(\theta) \, cos(2\, \theta)= \frac{1}{\pi} \int_0^{2\,\pi} \, d\theta (1 - sin(2\, \theta)) \, cos(2\, \theta)=0 \\
S_3&=B= \frac{2}{\pi} \int_0^{2\,\pi} \, d\theta \, I(\theta) \, sin(2\, \theta)= \frac{1}{\pi} \int_0^{2\,\pi} \, d\theta (1 - sin(2\, \theta)) \, sin(2\, \theta)=-1
\end{equation}
\section{Spin Injektion}
\label{Spininjection}
Die Überbevölkerung eines Spinzustandes durch externe Einflüsse wird als \textit{Spin Injektion} bezeichnet und kann über verschiedene Wege erreicht werden. Zunächst ist es möglich, einen ausgewählten Bandübergang durch polarisiertes Licht zu begünstigen. Beschränkt man sich energetisch auf die HH- und LH-Bänder, so wäre mit zirkulär polarisiertem Licht eine Polarisation $\text{P}=\frac{\text{3}-\text{1}}{\text{3}+\text{1}}=\text{50}\%$ ($P_\text{spin}=\text{2} \cdot \text{P}_\text{circ}$) zu erwarten. Durch die geringe Aufspaltung beider Bänder kommt es jedoch zu einem Überlapp beider strahlenden Übergänge, so dass Elektronen beider Bänder angeregt und, daraus resultierend, beide Polarisationszustände gleichzeitig besetzt werden. Für die maximale Polarisation gilt hier:
\begin{equation}
&a^2=\frac{\alpha_B}{\alpha_A}\\
&a=\frac{1}{x \cdot \sqrt{\frac{1}{x^2}+1}}  	&x=\frac{-(\Delta_1-\Delta_2)+\sqrt{(\Delta_1-\Delta_2)^2+8\cdot(\Delta_3)^2}}{2 \sqrt{2}\cdot \Delta_3}
\end{equation}
mit den Anregungswahrscheinlichkeiten der Valenzbänder A und B $\upalpha_\text{A,B}$, dem Kristall-Feld Parameter für ZnO $\Updelta_{\text{1}}=\text{43 meV}$ sowie der parallelen und orthogonalen Spin-Bahn-Parameter bezüglich der c-Achse $\Updelta_{\text{2}}=\Updelta_{\text{3}}=\Updelta_{\text{SO}}/3=\text{5.3 meV}$ \cite{Reynolds.1999} ergibt sich eine maximale Polarisation $\text{P}_{\text{max}}=\text{1.5}\%$. Somit ist eine optische Polarisation für unser Materialsystem quasi ausgeschlossen \cite[S. 299]{Morkoc.2009}, da auch eine lineare Polarisation des Lichtes beide Spinzustände gleichermaßen bevölkert.\\
\\
Magnetfelder bieten eine weitere Möglichkeit die Überpopulation eines Spinzustandes herbeizuführen. Durch die Zeeman-Aufspaltung \ref{Zee} wird die Entartung der beiden Spinzustände des Leitungsbandes aufgehoben. Entsprechend stellt sich das thermische Gleichgewicht zugunsten des energetisch abgesenkten Zustandes neu ein.\\
\\
Da elektrische Felder in dieser Thesis keine Verwendung finden, wird eine elektrische Spin Injektion hier nicht betrachtet, soll der Vollständigkeit halber aber erwähnt werden.