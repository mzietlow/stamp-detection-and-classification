\section{Introduction}
Digitalization is a recurring topic in German politics. Especially in governmental
contexts however, formal letter writing is still sometimes a requirement.

This paper is motivated by a use case identified within the author's corporation,
regarding incompletely filled out billing documents. While the respective document
contains rows explicitly for the name and address of the contractor, it is common
practice that contact details are not provided via the fields. Instead, contractors
use an ink stamp. Unfortunately, this practice excludes the document from currently
applied simple automation pipelines. Therefore, a multistage plan has been
forged, to enable automated document processing. Its first stage comprises the
detection of a \gls{bbox} around the stamp.

The paper is structured as follows. In \cref{sect:related-work}, related work
on the topic of stamp detection is collected and discussed. In \cref{sect:methodology},
the method that will be applied in this paper is thoroughly discussed. In \cref{sect:exp-setup},
a public dataset (\acrshort{staver}) for stamp detection is introduced, and results
are discussed. Finally, in \cref{sect:results-and-discussion}, the paper is wrapped
up and future research directions are presented.

Unfortunately, introduction to fundamental concepts from statistical learning is
beyond the scope of this paper. The reader is therefore expected to be roughly
familiar with basic terminology, regarding especially the usage of neural networks.
