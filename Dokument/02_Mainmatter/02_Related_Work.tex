\section{Related Work}\label{chap:Related Work}
In the past, a variety of methods for stamp detection have been proposed.
A patent on page segmentation, a closely related topic, dates back as far as
1985 on Google Patents. A typical strategy found in most publications is to
first separate a set of stamp-candidates from text and background, followed by
a verification of those candidates by applying thresholds on multiple
previously chosen features. To provide an overview of the topic, recent work on
stamp detection is split into two major- and several sub-categories.
\begin{description}
    \item [Restricted approaches]
        \begin{enumerate*}[label={\alph*)},font={\color{red!50!black}\bfseries}]
            \item [Color restricted]
                    In~\cite{Micenkova.2011091820110921},
                    ~\citeauthor{Micenkova.2011091820110921} present an
                    approach based on color clustering and geometric features.
                    To extract a set of stamp-candidates from a given image,
                    they assume stamps to be chromatic objects. In consequence,
                    all achromatic parts are dropped off the image. To separate
                    the remaining candidates for individual analysis, the
                    XY-Cut algorithm proposed in~\cite{Krishnamoorthy.1993} is
                    applied. For classification, an ensemble of geometric-
                    (pixel density, skew, etc.) and color-related features
                    (standard deviation of hue) is compared with empirically
                    set thresholds. By treating stamps as chromatic objects, 
                    they are unable to detect black stamps and face problems 
                    when colored fonts are used. In~\cite{Micenkova.2015},
                    \citeauthor*{Micenkova.2015} improve upon their method by
                    applying additional geometric features. Although they assume
                    applicability of the updated approach on black stamps,
                    this hypothesis is not verified.\\A predecessor of the
                    approach in~\cite{Micenkova.2011091820110921}
                    is~\cite{Forczmanski.2010} which employs less verification
                    features and is only applicable to blue or red stamps.\\
                \item [Geometrically restricted]
                    An early geometric analysis technique was formulated by
                    ~\citeauthor*{Zhu.2006} in~\cite{Zhu.2006}, utilizing a
                    novel ellipse detection algorithm inspired by Hough
                    transform. However, this method is solely suitable for
                    elliptical stamps.
        \end{enumerate*}
    \item [Generic approaches]
    \begin{enumerate*}[label={\alph*)},font={\color{red!50!black}\bfseries}]
        \item [Geometric features]
        A more versatile method than~\cite{Zhu.2006} was proposed in
        \cite{Ahmed.2013082520130828, Ahmed.2016}.
        \citeauthor*{Ahmed.2013082520130828} make use of advances in
        feature detectors and feature descriptors. In a first step, the
        image is binarized and connected components are extracted. For each
        connected component, feature key points and feature descriptors are
        computed via the FAST and ORB algorithms. In addition, bounding box
        height and width of each connected component are extracted.
        Verification is performed by comparing the extracted features of each
        connected component to a (remarkably) small training set. However,
        ~\citeauthor*{Ahmed.2013082520130828} notice low precision-
        and recall-rates, caused by their inability to recognize severely
        overlapping stamps.\\
        In~\cite{Nandedkar.2015121620151219},
        \citeauthor*{Nandedkar.2015121620151219} extend an approach they
        suggested in~\cite{Nandedkar.2015082320150826}. They analyze spatial
        frequency components in document images, finding text to be the major
        contributor. To segment a given image, they first conceal
        text-components by removing objects with high spatial frequencies and
        further apply a Gaussian filter. To prevent textual stamps from being
        removed, a chromaticity threshold is applied on the removal of spatial
        frequency components. To identify stamp-candidates, mean-shift
        segmentation is applied. Small regions are filtered with an empirically
        set threshold, the largest region is considered to be the background.
        For verification of stamp-candidates, AlexNet is used, producing three
        classes (logo, stamp and noise) in the final \gls{dense layer}. Even 
        though unmentioned in the paper, the presented approach of filtering 
        spatial frequency components faces limitations regarding black textual 
        stamps which might turn out to be removed by the spatial frequency 
        filter.\\
        \item [Mixed features]
        \citeauthor*{Dey.2015121620151219} propose an outlier-based
        method in~\cite{Dey.2015121620151219}. To segment a given image,
        it is separated in foreground and background using binarization. To
        retain color information the obtained foreground-values are applied as
        a mask to the original image. Color information is reduced from
        three-dimensional \(RGB\) color space to a one-dimensional principal
        component using the Principal Component Algorithm (\(PCA\)).
        Stamp-candidates are obtained using a connected components algorithm on
        the first principal component. To filter out false positives,
        thresholds on features such as stroke width, bounding box width and
        height are applied. For verification a set of features is extracted and
        validated using additional empirically set thresholds.\\

        A sliding window approach using the AdaBoost algorithm on cascaded
        decision trees has been proposed in~\cite{Forczmanski.2016}. To train 
        AdaBoost, a labeled training set is constructed. For each training 
        sample, Haar-like features are extracted. AdaBoost then constructs a 
        set of consecutive decision trees to optimize stamp classification of 
        training samples. Each decision tree takes in a Haar-like feature from 
        a fixed position inside the sliding window. Based on the input feature 
        value the tree decides whether to accept or reject a given window as 
        stamp-candidate. A set of consecutive decision trees with such property 
        (accept or completely reject) is called a cascade. In order for a 
        sliding window to become a valid stamp-candidate it has to survive this 
        cascade of decision trees. The set of generated candidates is then 
        verified using a broader set of features. Several algorithms are 
        compared for verification including 1NN, decision trees, and \glspl{svm}.\\

        In~\cite{Younas.2017110920171115}, \citeauthor*{Younas.2017110920171115}
        propose an approach using a \gls{fcn}, built upon the VGG-16 network
        architecture. In this approach, the final \glspl{dense layer} of VGG-16
        are replaced with a \gls{deconvolutional layer}. By doing this, a
        prediction map is received as output instead of classification scores.
        To reduce training time, transfer learning from the VOC2011 dataset is
        applied. However, no preprocessing is mentioned. Using 90\% of their
        dataset for training and 10\% for verification they report 
        state-of-the-art results, although noticing difficulties with tabular
        stamps.
    \end{enumerate*}
\end{description}

Despite numerous research carried out on stamp segmentation, most of the 
approaches examined in this chapter either face limitations with overlapping 
objects~\cite{Nandedkar.2015082320150826, Nandedkar.2015121620151219, 
Forczmanski.2016, Ahmed.2013082520130828, Dey.2015121620151219, 
Forczmanski.2015, Forczmanski.2016} or logo misclassification~\cite
{Ahmed.2013082520130828, Dey.2015121620151219,Micenkova.2011091820110921}. Some
approaches are limited by runtime, operating for more than 13 Seconds per
evaluation~\cite{Ahmed.2016,Nandedkar.2015082320150826}.
\par
Besides segmentation, classification of stamps is observed to be a common task 
in literature. Classes are usually either binary (Stamp and non-Stamp), 
object-related (Stamp, Logo, Text, Tables)~\cite{Forczmanski.2016,
Nandedkar.2015082320150826, Nandedkar.2015121620151219, Dey.2015121620151219},
or shape-dependent (Circle, Ellipse, Square, etc.)~\cite{Forczmanski.2015}.
However, research on classification of similar stamps i.e.\ stamps belonging to
the same shape-class but differing in text and decor, has so far only been
reported on by~\cite{Petej.2013070720130710}.