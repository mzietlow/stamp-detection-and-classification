\chapter{Fazit}
Diese Arbeit gibt eine Übersicht, über die zwei grundlegenden OAuth Grant Types,
Authorization Code Grant und Implicit Grant. Es wurde diskutiert, wie OAuth
trotz unsichere User-Agents implementiert werden kann. Durch Betrachtung
beispielhafter Flow-Diagramme wurde die Übermittlung von Tokens im Klartext
ausgeschlossen. Aufgrund der besonderen Situation, dass die Websession zu Beginn
nicht auf den Nutzer bezogen werden kann, wurden auch MAC-Tokens verworfen. Die
Übermittlung wurde schließlich durch asymmetrisch verschlüsselte JWE Container
gelöst. Von der Übermittlung eines unverschlüsselten Access-Tokens ist in jedem
Fall abzuraten, wie in Kapitel~\ref{ch:entkoppelteFlows} dargelegt wurde. Eine
Kopplung zwischen App- und Web-Client, wie in Kapitel~\ref{ch:GekoppelteFlows}
vorgeschlagen, ist zu empfehlen, da sie das Risiko, das von entwendeten
Parametern ausgeht, minimiert. Eine Verbesserung würde die Erweiterung (wie in
Abschnitt~\ref{ch:entkoppelteFlows}) des vorgeschlagenen Flows
(Abbildung~\ref{ls: Extended Implicit Authorization TK}) auf einen Authorization
Code Grant bieten, da Authorization Codes nur einmalig genutzt werden können.
Aufgrund des erhöhten Komplexität wurde davon in dieser Arbeit letztlich
abgesehen.
