\documentclass[12pt,oneside,ams,a4paper]{hepthesis}
\usepackage{polyglossia}
\setmainlanguage[babelshorthands=true]{german}
\setotherlanguage[variant=british]{english}
\usepackage{pdfpages}
\usepackage{ulem} % Unterstreichen von Text
\usepackage{hyperref}
\usepackage{nameref}
\usepackage[section]{placeins} % stoppt Floats an Section-Grenzen
\KOMAoption{numbers}{noenddot}
\addtokomafont{labelinglabel}{\sffamily}
\usepackage{booktabs} % Erweiterung von Tabellen
\usepackage{datetime}
\usepackage{fontspec}
\usepackage{titling}
\usepackage{glossaries} % \gls{<golssary-entry>} makeglossaries <dateiname> per cmd
\usepackage{fixltx2e}
\usepackage{blindtext}
\usepackage{geometry}
\usepackage{setspace}
\usepackage{listings}
\usepackage{color}
\usepackage{nameref}
\usepackage{todonotes}

\makeglossaries{}
%\setstretch{1.0269} % 1.25-facher Zeilenabstand bei 12pt.
\spacing{1.25}
\geometry{a4paper,left=20mm,right=20mm, top=30mm, bottom=20mm}
\author{Malte Zietlow}
\title{Analyse von OAuth 2.0 und OpenID Connect zur Weiterreichung der
\mbox{zwei-Faktor Authentifizierung} aus der TK-App an Meine~TK.}
\date{\today}
\hypersetup{
 pdfauthor = {\theauthor},
 pdftitle = {\thetitle},
 pdfsubject = {Transferleistung, Techniker Krankenkasse/Nordakademie, 2018},
 bookmarksopen=true
}
\raggedbottom{}
\setmainfont[Path=C:/Windows/Fonts/]{times.ttf}

\definecolor{lightgray}{gray}{0.9}

\lstset{
    showstringspaces=false,
    basicstyle=\ttfamily,
    keywordstyle=\color{blue},
    commentstyle=\color[grey]{0.6},
    stringstyle=\color[RGB]{255,150,75}
}

\begin{document}
\newcommand{\theLocationAndDate}{Hamburg, den \thedate}
\newcommand{\inlinecode}[2]{\colorbox{lightgray}{\lstinline[language=#1]$#2$}}
\renewcommand*\chapterheadstartvskip{\vspace*{-1cm}}
\renewcommand*{\figurename}{Abb.}
\renewcommand*{\figureautorefname}{Abb.}
\renewcommand*{\tablename}{Tab.}
\renewcommand*{\tableautorefname}{Tab.}
\renewcommand*{\chapterautorefname}{Kapitel}
\renewcommand*{\sectionautorefname}{Kapitel}
\renewcommand*{\subsectionautorefname}{Kapitel}
\renewcommand*{\equationautorefname}{Gl.}

\setacronymstyle{long-short-desc}
\newacronym[description={Organisation von
Workinggroups, die sich mit je einem spezifischen Thema befassen. Teil der
Internet Society (ISOC)}]{ietf}{IETF}{Internet Engineering Task Force}
\newacronym[description={Hostet Nutzerinformationen}]{idp}{IdP}{Identity
Provider}


\includepdf[pages=1]{misc/deckblattTerminiert.pdf}
\newpage
\begin{frontmatter}
    % Klassische Titelseite

\thispagestyle{empty}
\begin{center}
\vspace*{-2cm}
\includegraphics[width=0.85\textwidth]{Bilder/Logo_FSU}\\
\vspace*{3cm}
    {\titlefont \huge \onehalfspacing
	\thetitle 
    \par}%
   \vfill
  %\vskip 6em
    {\normalfont\normalcolor\bfseries
	\large
	Masterthesis \\
	\large
	im Studiengang Physik\\
	angefertigt am Institut für Festkörperphysik\\
	der Physikalisch-Astronomischen Fakultät\\
	der Friedrich-Schiller-Universität Jena
    \par}%
\end{center}\par
%\vfill
\vspace*{2.5cm}
\noindent\begin{minipage}[b]{\textwidth}
{
  \noindent \textbf{von B.Sc. Christian Zietlow, geb.~am 15.~Mai 1992 in Bad Oldesloe}\\

  \begin{tabbing}
  \textbf{Betreuer und 1.~Pr\"ufer :}  \= \textbf{Prof. Dr. Carsten Ronning, Friedrich-Schiller-Universität Jena}\\
  \textbf{\qquad \qquad \qquad \, 2.~Pr\"ufer :} \> \textbf{Dr. Claudia Schnohr, Friedrich-Schiller-Universität Jena} \\
  \end{tabbing}

  \noindent \textbf{Jena, den  01.~August 2018}
  }
\end{minipage}



\thispagestyle{empty}
\addtocounter{page}{-3}

\begin{abstract}[Kurzzusammenfassung]
    \thispagestyle{plain} \addtocounter{page}{-1} \fussy
    Lorem ipsum dolor sit amet, consetetur sadipscing elitr, sed diam nonumy
    eirmod tempor invidunt ut labore et dolore magna aliquyam erat, sed diam
    voluptua. At vero eos et accusam et justo duo dolores et ea rebum. Stet
    clita kasd gubergren, no sea takimata sanctus est Lorem ipsum dolor sit
    amet. Lorem ipsum dolor sit amet, consetetur sadipscing elitr, sed diam
    nonumy eirmod tempor invidunt ut labore et dolore magna aliquyam erat, sed
    diam voluptua. At vero eos et accusam et justo duo dolores et ea rebum. Stet
    clita kasd gubergren, no sea takimata sanctus est Lorem ipsum dolor sit
    amet. Lorem ipsum dolor sit amet, consetetur sadipscing elitr, sed diam
    nonumy eirmod tempor invidunt ut labore et dolore magna aliquyam erat, sed
    diam voluptua. At vero eos et accusam et justo duo dolores et ea rebum. Stet
    clita kasd gubergren, no sea takimata sanctus est Lorem ipsum dolor sit
    amet.

    Duis autem vel eum iriure dolor in hendrerit in vulputate velit esse
    molestie consequat, vel illum dolore eu feugiat nulla facilisis at vero eros
    et accumsan et iusto odio dignissim qui blandit praesent luptatum zzril
    delenit augue duis dolore te feugait nulla facilisi. Lorem ipsum dolor sit
    amet, consectetuer adipiscing elit, sed diam nonummy nibh euismod tincidunt
    ut laoreet dolore magna aliquam erat volutpat.

    Ut wisi enim ad minim veniam, quis nostrud exerci tation ullamcorper
    suscipit lobortis nisl
\end{abstract}
\setlength{\evensidemargin}{0cm} %Zum Drucken -0.6cm Rand einstellen!
\setlength{\oddsidemargin}{0cm}	 %Zum Drucken 0.6cm Rand einstellen!
\tableofcontents \newpage \thispagestyle{empty}
%\listoffigures
%\listoftables

\end{frontmatter}

\begin{mainmatter}
    \chapter{Einleitung} In der heutigen Hochtechnologiegesellschaft besteht auch
weiterhin ein \mbox{ungebremstes} Interesse daran, immer schnellere, kleinere
und effizientere Prozessoren zu \mbox{entwickeln}. Bisher konnte die
Halbleiterindustrie diesem weitestgehend nachkommen, dennoch zeichnen sich
Grenzen dieser Miniaturisierung ab, die unüberwindbar scheinen. Grundlegende
physikalische Begebenheiten wie Quanteneffekte führen dazu, dass funktionale
Strukturen nicht weiter verkleinert werden können, ohne sie massiv zu
beeinträchtigen und damit unbrauchbar werden zu lassen~\cite{Moore.2017}.\\  Um
den technischen Fortschritt auch in Zukunft vorantreiben zu können, bedarf es,
ob der Unumgänglichkeit dieser Effekte, neuer Denkansätze. Einer dieser Ansätze
umfasst die Verwendung optischer Schaltkreise. Diese bieten den Vorteil eines
Informationstransportes in Lichtgeschwindigkeit, im Gegensatz zu der deutlich
langsameren \mbox{Geschwindigkeit} der Elektronen in konventionellen
Schaltkreisen. Auch die Schaltzeiten sind gegenüber den herkömmlichen
elektronischen Bauteilen stark reduziert~\cite{Simonite.2010,Johnson.2015}, denn
bei elektronischen Bauteilen ist die Driftgeschwindigkeit der Elektronen ein
limitierender Faktor – dieser entfällt bei optischen Bauteilen. Durch beide
\mbox{Begebenheiten}, der schnelleren Transportgeschwindigkeit und den kürzeren
Schaltzeiten, \mbox{erhöht} sich der Datentransfer gegenüber herkömmlichen
elektronischen Prozessoren drastisch. Die Integrierte Optik (IO) als Technik der
nächsten Generation steckt aber noch in den Kinderschuhen und es wird ausgiebig
an der Verwirklichung geforscht~\cite{Touch.2017}, in Erwartung, dass der Markt
in den nächsten Jahren stark wachsen wird~\cite{Credence.2017}.\\ Für diese neue
Art des Rechnens wird eine gerichtete kohärente Lichtquelle benötigt – ein
Laser. Halbleiternanodrähte bilden eine solche Klasse miniaturisierter Laser und
könnten in Zukunft diese Lücke füllen. Sie besitzen einen Durchmesser von
wenigen hundert Nanometern, bei einer Länge von wenigen Mikrometern. Aufgrund
eines \mbox{Brechungsindex} größer eins, können sie Licht einer Wellenlänge
führen, die größer ist als ihr eigener Durchmesser~\cite{Zimmler.2010}. Damit
setzen Nanodrähte eine untere Grenze für die gerichtete Laseremission,
gleichzeitig fungieren sie aufgrund ihrer Struktur als natürliche Resonatoren
~\cite{Eichhorn.2013}. Sie besitzen Schaltzeiten im Bereich weniger
Pikosekunden, drei Größenordnungen unter den Schaltzeiten moderner
Transistoren~\cite{Sidiropoulos.2014,Qiu.2017}. Damit erfüllen Nanodrähte die
Anforderungen, die die Gesellschaft an ein solches System stellt – sie sind
schnell, sie sind klein und sie sind effizient.

\input{Dokument/Analyse/OAuth/OAuth}
\chapter{OpenID Connect}
\blindtext{}
\input{Dokument/Analyse/OpenID/Grundlagen.tex}
\input{Dokument/Analyse/OpenID/Flows.tex}
\input{Dokument/Analyse/OpenID/Sicherheitsanalyse.tex}
\input{Dokument/Analyse/OpenID/Fazit.tex}

\chapter{Evaluation}
\blindtext{}

\input{Dokument/Verdict.tex}
\chapter{Implementation} Der in unserem Team aufgetretene Anwendungsfall
involviert zwei Clients, die TK-App und den Web-Client, mit jeweils unabhängigen
Backendsessions. Die TK-App ist als sicherer Client zu betrachten, der
Web-Client kommuniziert per HTTPS und nutzt secured-Cookies. Durch seine
Backendanbindung ist auch er als sicher einzustufen. Bei erstmaligem Aufruf des
Web-Clients wird im User-Agent des Nutzers ein neuer, noch nicht eingeloggter
TKSESSION-Cookie gesetzt. Es wird angenommen, dass der Resource Owner bereits in
der App eingeloggt ist. Zielsetzung ist nun, die im User-Agent des Web-Clients
gesetzte Session einzuloggen. Zur Verfügung stehen die bereits aufgezählten
Parteien: eingeloggte TK-App + Backend, User-Agent, Web-Client + Backend.
\\
In ersten Überlegungen wurde die Erstellung einer neuen, eingeloggten Session
in Betracht gezogen, was sich jedoch schnell als Vergehen am Loadbalancer
herausstellte und daher verworfen wurde. Auch der Ansatz, die Session ID an die
TK-App weiterzuleiten und in ihrem Backend einzuloggen erwies sich als
undurchführbar. Um eine Session einzuloggen, muss ein Session-eigener
User-Context befüllt werden, was durch eine außerhalb liegende Session unmöglich
ist. \todo{redirectAPI erklären.}
\\
Die App selbst ist also nicht dazu in der Lage, die Web-Session einzuloggen.
Nur der Web-Client selbst kann dies bewerkstelligen. Durch diese Erkenntnis kann
auf eine klassische \gls{OAuth}-Problemstellung reduziert werden: der Web-Client
greift auf eine Login-API zu. Um diese vor unbefugten Nutzern zu schützen,
setzt der Zugriff auf sie einen Access-Token voraus. Auch lässt sich in der TK-
Implementation von dem Access-Token auf den Versicherten schließen, für den er
ausgestellt wurde, ein weiterer Parameter ist also nicht erforderlich.
\section{Implicit Grant-Login}

Abbildung~\ref{ls: Implicit Authorization TK} zeigt eine mögliche Realisierung
als von der App ausgehender Implicit Flow. Sie ist die einfachste Realisierung
des Anwendungsfalls. \todo{Ausführen, weshalb einfach. Bloß ein Web-Api-Call}
Auf den zweiten Blick, weist dieses Vorgehen aber zwei wichtige Fehler auf:
zunächst dies, die einzige Verbindung zwischen App- und Web-Client, ist der in
Schritt 3 übertragene Access-Token. Web-APIs können jedoch von jedem User-Agent
aufgerufen werden. Wir sollten also davon ausgehen, dass unsere API von Dritten
angegriffen --- und diese Angriffe erfolgreich sein werden, falls es einem
Dritten gelingen sollte, in den Besitz eine Access Tokens zu gelangen. Hier
also, ebenfalls in Schritt 3, findet sich der zweite Fehler. Die Übertragung des
Access Tokens an den Web-Client erfolgt unverschlüsselt als Query Parameter.
Durch einen korrumpierten User Agent, respektive Smartphone, könnte es einem
Dritten nun möglich sein, den Access Token zu entwenden. Da der Token
unverschlüsselt übermittelt wurde, kann der Dritte sich durch einen Aufruf
der login/redirect-API Zugang zu einer eingeloggten Session verschaffen.

\begin{figure}[h]
    \scalebox{.6}{
        \lstinputlisting[linewidth=25cm]{Dokument/TkFlows/implicit.ascii}
    }
    \caption{Anwendungsfall, Implicit Login Flow}\label{ls: Implicit Authorization TK}
\end{figure}


Ist ein Pingpong _wirklich_ nötig oder reicht ein Session JWE?

\chapter{Fazit}
Diese Arbeit gibt eine Übersicht, über die zwei grundlegenden OAuth Grant Types,
Authorization Code Grant und Implicit Grant. Es wurde diskutiert, wie OAuth
trotz unsichere User-Agents implementiert werden kann. Durch Betrachtung
beispielhafter Flow-Diagramme wurde die Übermittlung von Tokens im Klartext
ausgeschlossen. Aufgrund der besonderen Situation, dass die Websession zu Beginn
nicht auf den Nutzer bezogen werden kann, wurden auch MAC-Tokens verworfen. Die
Übermittlung wurde schließlich durch asymmetrisch verschlüsselte JWE Container
gelöst. Von der Übermittlung eines unverschlüsselten Access-Keys ist in jedem
Fall abzuraten. Eine Kopplung zwischen App- und Web-Client, wie in
Kapitel~\ref{ch:GekoppelteFlows} vorgeschlagen, ist zu empfehlen, da sie das
Risiko, das von entwendeten Parametern ausgeht, minimiert. Eine Verbesserung
würde die Erweiterung des vorgeschlagenen Flows (Abbildung~\ref{ls: Extended
Implicit Authorization TK}) auf einen Authorization Code Grant bieten. Aufgrund
der erhöhten Komplexität wurde davon in dieser Arbeit letztlich abgesehen.


\end{mainmatter}

\begin{appendices}
    \chapter{Anhang}
\begin{figure}[h]
    \scalebox{.6} {
        \lstinputlisting[nolol]{Dokument/JWECalc/main.ascii}
    }
    \caption{main-Methode zur ver- und entschlüsselung eines JWEs in Java mit nimbusds}\label{fig: main}
\end{figure}

\begin{figure}
    \scalebox{.6} {
        \lstinputlisting[nolol]{Dokument/JWECalc/keyGenerator.ascii}
    }
    \caption{Asymmetrischer Key Generator, RSA}\label{fig: keyGenerator}
\end{figure}

\begin{figure}
    \scalebox{.6} {
        \lstinputlisting[nolol]{Dokument/JWECalc/jweGenerator1.ascii}
        }
    \caption{Klasse zur ver- und entschlüsselung eines JWE-Tokens mit
    nimbusds}\label{fig: jweDecrypt}
\end{figure}
\begin{figure}[t]
    \scalebox{.6} {
        \lstinputlisting[nolol]{Dokument/JWECalc/jweGenerator2.ascii}
        }
    \caption{Klasse zur ver- und entschlüsselung eines JWE-Tokens mit
    nimbusds}\label{fig: jweEncrypt}
\end{figure}

\end{appendices}

\begin{backmatter}
    \addtocontents{toc}{\vspace*{1em}}
% Literaturliste aus Literaturdatenbank (Bibtex-Datei) bauen. Nur tatsaechlich
% zitierte Literatur wird in Liste aufgenommen.

% Regelmaessigen Aufruf von ``bibtex abschlussarbeit'' nicht vergessen, falls
% das der Latex-Editor nicht erledigt.

% Zur Bearbeitung der Literaturdatenbank kann das Programm JabRef
%(http://jabref.sf.net) empfohlen werden. (Java-Programm, laeuft unter Windows,
% Linux, Mac, \ldots)

\bibliography{Literatur/bibtex/Literatur}
% Stil fuer Literaturliste festlegen
% Variante A: DIN, Eintraege erhalten Kuerzel aus Autoren-Initialien und Jahr,
% alphabetisch geordnet

            %\bibliographystyle{alphadin}

% Variante B: DIN, Eintraege werden durchnummeriert, alphabetisch geordnet

            %\bibliographystyle{unsrt}

% Variante C: Harvard.
             \bibliographystyle{agsm}

\newpage
\thispagestyle{empty}
\section*{Selbständigkeitserklärung}
Ich versichere hiermit an Eides statt, dass ich die vorliegende Arbeit
selbstständig angefertigt und ohne fremde Hilfe verfasst habe, keine außer den
von mir angegebenen Hilfsmitteln und Quellen dazu verwendet habe und die den
benutzten Werken inhaltlich und wörtlich entnommenen Stellen als solche
kenntlich gemacht habe. \vspace*{2cm}

\begin{flushright} \theLocationAndDate{} \end{flushright}

\end{backmatter}

\end{document}
